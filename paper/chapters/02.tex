\chapter{A local-global principle.}
\label{chap:local-global-principle}

{\scshape In this} chapter we shall prove the Hasse-Minkowski theorem for quadratic forms over \(\Rationals\). We shall follow Chapters 2\,--\,4 of \cite{serre2012course}, introducing and developing the structure theorems for the ring \(\Integers_p\) of \(p\)-adic integers and the field \(\Rationals_p\) of \(p\)-adic numbers together with the main result from Hensel on the roots of polynomials over \(\Integers_p\), followed by the introduction of the Hilbert symbol and some of its properties, and culminating in the proof of the Hasse-Minkowski theorem.

\section{The field of \(p\)-adic numbers.}

\subsection{}~The history of the theory of quadratic forms over \(\Rationals\) owes much of its development to the theory of \(p\)-adic numbers as introduced by Hensel and expanded by his student Hasse.\,\cite{hasse1922uber,hensel1913zahlentheorie} In this section, we shall introduce the field of \(p\)-adic numbers and prove some of its basic properties and develop the results we need as they relate to quadratic forms. For a more detailed treatment of the theory of \(p\)-adic numbers, we refer the reader to \cite{gouvea1997p,koblitzp}. The \(p\)-adic field can be constructed both analytically and algebraically; our approach in this paper shall be mainly algebraic.\label{sec:field-of-p-adic-numbers}

% The theory of \(p\)-adic numbers has been motivated historically by the desire to find the solutions to congruences of the form
% \begin{equation}
%     x^2 \equiv a \pmod{p^n}
%     \label{eq:congruence-mod-pn}
% \end{equation}
% for a prime number \(p\) and an integer \(a\) and for all \(n \geq 1\). \cite{amice1975nombres} Consider, for example, the case \(p = 5\) and \(a = -1\). For \(n = 1\), this can be simplified to \(x^2 \equiv -1 \pmod{5}\), which has the solutions \(x \equiv 2\) and \(x \equiv -2 \pmod{5}\). Observe that for any \(k \leq n\)

% \subsection{}~
We begin with a formal definition, following \cite{serre2012course}. Let \(A_n := \Integers/p^n \Integers\) be the set of equivalence classes of integers modulo \(p^n\) for a prime number \(p\) for each \(n \in \Naturals\). Each \(A_n\) is a ring and induces a homomorphism\label{sec:padic-def}
\[
  \phi_n : A_{n} \to A_{n - 1}  
\]
which is surjective and has kernel \(p^{n-1}A_n\). Define the sequence of rings and the corresponding homomorphisms
\begin{equation*}
    \dots \to A_n \xrightarrow{\phi_n} A_{n - 1} \xrightarrow{\phi_{n - 1}} \dots \xrightarrow{\phi_2} A_1
\end{equation*}
indexed by \(\Naturals\). The \emph{ring of \(p\)-adic integers} \(\Integers_p\) is defined to be the inverse limit of this sequence, that is,
\[
  \Integers_p := \varprojlim A_n = \left\{
    (a_n)_{n \in \Naturals} \in \prod_{n \in \Naturals} A_n : \phi_n(a_n) = a_{n - 1} \text{ for all } n \in \Naturals
  \right\}.
\]
We shall see shortly that this is indeed a ring. Each element of \(\Integers_p\) is a sequence of the form \((\dots, x_3, x_2, x_1)\) and thus it is often useful to think of an element \(\alpha\) of \(\Integers_p\) as a formal sum
\[
  \alpha = a_0 + a_1p + a_2p^2 + \cdots
\]
Addition and multiplication in \(\Integers_p\) are defined componentwise, that is, if \(\alpha = (a_n)_{n \in \Naturals}\) and \(\beta = (b_n)_{n \in \Naturals}\) are elements of \(\Integers_p\), then \(\alpha + \beta = (a_n + b_n)_{n \in \Naturals}\) and \(\alpha\beta = (a_nb_n)_{n \in \Naturals}\). Elements of \(\Integers_p\) are called \emph{\(p\)-adic integers}.

We proceed by exploring some properties of \(\Integers_p\). First fix a natural number \(n\) and define the map \(\epsilon_n: \Integers_p \to A_n\) which associates with every \(p\)-adic integer \(\alpha\) its \(n\)-th component \(a_n\). Then we have
\begin{theoremx}
    The sequence \(0 \to \Integers_p \xrightarrow{p^n} \Integers_p \xrightarrow{\epsilon_n} A_n \to 0\) is a short exact sequence of abelian groups.
\end{theoremx}

See \cite[pp.~11--12]{serre2012course} for a proof.

\begin{theoremx}\label{thm:criterion-units-of-zp}
    An element of \(\Integers_p\) is invertible if and only if it is not divisible by \(p\).
\end{theoremx}

\begin{proof}
    Let \(\alpha = (a_n) \in \Integers_p.\) If \(p\) does not divide \(\alpha\) then \(p\) does not divide each of the \(a_n\) and in particular \(a_1 \not\equiv 0 \pmod{p}\). By the Euclidean algorithm, there exists integers \(\xi\) and \(\eta\) such that \(a_1\xi + p\eta = 1\). Because \(p \mid p\eta\), it follows that \(a_1\xi \equiv 1 \pmod{p}\). Let \(z = 1 - \xi\alpha\); we can then compute the inverse of \(\alpha\) as follows:
\end{proof}

\begin{theoremx}\label{thm:units-of-zp}
    Let \(\Units\) be the set of invertible elements of \(\Integers_p\). Then every element of \(\Integers_p\) can be written uniquely as \(\alpha = p^nu\) where \(n \geq 0\) and \(u \in \Units\).
\end{theoremx}

\begin{proof}
    If \(\alpha \in \Integers_p\) is not zero, then there exists a largest integer \(n\) such that \(a_n = \epsilon_n(a) = 0\); then \(\alpha = p^nu\) where \(u\) is not divisible by \(p\) (and hence by Theorem \ref{thm:criterion-units-of-zp} is invertible). That \(n\) is unique follows from the fact that \(p^n\) is injective.
\end{proof}

\smallskip

The unique integer \(n\) defined in Theorem \ref{thm:units-of-zp} induces a function \(\nu_p : \Integers_p \to \Integers_{\geq 0}\), i.e., we shall write \(\nu_p(\alpha) = n\) where \(\alpha = p^nu\) for some \(u \in \Units\). This function is called the \emph{\(p\)-adic valuation} of \(\alpha\). We write \(\nu_p(0) = +\infty\), and the reader may verify that it satisfies the following properties for all \(\alpha, \beta \in \Integers_p\):

\smallskip

\begin{enumerate}[nosep, label=(\roman*)]
    \item \(\nu_p(\alpha\beta) = \nu_p(\alpha) + \nu_p(\beta)\); and
    \item \(\nu_p(\alpha + \beta) \geq \min\{\nu_p(\alpha), \nu_p(\beta)\}\).
\end{enumerate}

\smallskip

\noindent From this we conclude that \(\Integers_p\) has no zero divisors and is therefore an integral domain. The valuation \(\nu_p\) also allows us to define the \(p\)-adic absolute value \(|\ |_p : \Integers_p \to \Reals_{\geq 0}\) by
\[
  |\alpha|_p = p^{-\nu_p(\alpha)},
\]
for all \(\alpha \in \Integers_p\), with equality if and only if \(\alpha = 0\). The \(p\)-adic absolute value induces the distance function
\[
  d(\alpha, \beta) := |\alpha - \beta|_p.
\]
The reader may verify that this is indeed a metric on \(\Integers_p\). Finally, the ring \(\Integers_p\) is complete with respect to this metric and \(\Integers\) is dense in \(\Integers_p\). \cite[p.~12]{serre2012course}

Because \(\Integers_p\) is an integral domain, we can define the field of fractions of \(\Integers_p\), which we shall denote by \(\Rationals_p\). The elements of \(\Rationals_p\) are called \emph{\(p\)-adic numbers}.

\subsection{Polynomials over \(\Integers_p\) and \(\Rationals_p\).}~We shall now consider polynomials with coefficients in \(\Integers_p\). We are interested, as usual, in finding the roots of some polynomial \(f(x)\) in \(\Integers_p[x]\), i.e., we want to find the \(p\)-adic integers (or \(p\)-adic numbers) \(\alpha\) such that \(f(\alpha) = 0\). We can observe, however, that finding a solution in \(\Rationals_p\) for some prime \(p\) does not necessarily guarantee the existence of solutions in \(\Rationals\). For example, the polynomial \(x^2 - 2\) has roots in \(\Rationals_7\) but clearly \(\sqrt{2} \notin \Rationals\). We shall defer answering the question of when we can find solutions in \(\Rationals\) to later in this chapter; for now we shall investigate the existence of solutions in \(\Integers_p\) and \(\Rationals_p\).\label{sec:polynomials-over-zp}

We shall again denote by \(A_n\) the ring \(\Integers / p^n \Integers\) of integers modulo \(p^n\). If \(f\) is a polynomial in \(\Integers_p[x]\) then for any \(n \geq 1\), we write \(f_n\) for the reduction of \(f\) modulo \(p^n\), i.e., \(f_n(x) = f(x) \pmod{p^n}\).

We begin with a lemma.

\begin{lemma}
    If \(\dots \to D_n \to D_{n - 1} \to \dots \to D_1\) is a projective system (as in {\normalfont \S\,\ref{sec:padic-def}}) and \(D = \varprojlim D_n\), then \(D\) is nonempty if each \(D_n\) is nonempty.
\end{lemma}

\emph{Proof.} See \cite[p.~13]{serre2012course}.

\begin{theoremx}\label{thm:zeros-of-polynomials-in-zp}
    Let \(\mathfrak{F}\) be a family of polynomials in \(m\) variables with coefficients in \(\Integers_p\). The following statements are equivalent:

    \smallskip

    \begin{enumerate}[nosep, label=(\alph*)]
        \item All the polynomials in \(\mathfrak{F}\) have a common root in \(\Integers_p^m\).
        \item For all \(n > 1\), all the polynomials \(f_n\) for each \(f \in \mathfrak{F}\) have a common root in \(A_n^m\).
    \end{enumerate}
\end{theoremx}

\begin{proof}
    Let \(R\) (resp. \(R_n\)) be the set of common roots of the polynomials in \(\mathfrak{F}\) in \(\Integers_p^m\) (resp. \(A_n^m\)). The \(R_n\) are finite and we have \(R = \varprojlim R_n\). The theorem follows from the lemma.
\end{proof}

We say that an element \(x = (x_1, \dots, x_m)\) of \(\Integers_p^m\) is primitive if at least one of the \(x_i\) is invertible, i.e., not all of the \(x_i\) are divisible by \(p\). We define similarly the notion of a primitive element of \(A_n^m\). We now relate Theorem \ref{thm:zeros-of-polynomials-in-zp} to solutions in \(\Rationals_p\).

\begin{theoremx}
    Let \(\mathfrak{F}\) be a family of homogeneous polynomials in \(m\) variables with coefficients in \(\Integers_p\). Then the following statements are equivalent:

    \smallskip

    \begin{enumerate}[nosep, label=(\alph*)]
        \item The polynomials in \(\mathfrak{F}\) have a nontrivial common root in \(\Rationals_p^m\).
        \item The polynomials in \(\mathfrak{F}\) have a common primitive root in \(\Integers_p^m\).
        \item For all \(n > 1\), the polynomials \(f_n\) for each \(f \in \mathfrak{F}\) have a common primitive root in \(A_n^m\).
    \end{enumerate}
\end{theoremx}

\begin{proof}
    The equivalence (b) \(\iff\) (c) follows from the lemma above. We prove (a) \(\iff\) (b). If \(x\) is a common primitive root of the polynomials in \(\mathfrak{F}\) in \(\Integers_p^m\), then \(x \neq 0\) and is also a common root of the polynomials in \(\mathfrak{F}\) in \(\Rationals_p^m\). Conversely, let \(x = (x_1, \dots, x_m)\) be a nontrivial common root of the polynomials in \(\mathfrak{F}\) in \(\Rationals_p^m\) and write
    \[
      h = \inf\{\nu_p(x_i) : 1 \leq i \leq m\}.
    \]
    Let \(y = p^{-h}x\) so that \(y\) is a primitive element of \(\Integers_p^m\). Moreover, \(y\) is a common root of the polynomials in \(\mathfrak{F}\) in \(\Integers_p^m\).
\end{proof}

The theorems in this section can be used to show that a polynomial in has a root in \(\Integers_p\) by proving that it has a solution in \(A_n\). \cite{weismann2006annotations}


\subsection{Hensel's lemma.}~We now introduce an important result, due to Hensel, which allows us to pass from a solution modulo \(p^n\) to a ``true'' solution in \(\Integers_p\). In essence, this result provides a meaningful way to approximate roots of polynomials in \(\Integers_p[x]\). Finding the root of a polynomial modulo \(p^n\) is the same as finding a value which is within the radius \(p^{-n}\) of the solution in the \(p\)-adic metric \cite{weismann2006annotations} and thus the goal of this approximation algorithm is to make \(n\) successively larger (and thus the radius \(p^{-n}\) successively smaller) until we have a solution in \(\Integers_p\). This result, which we shall call Hensel's lemma, is similar to Newton's method in calculus for finding roots of polynomials given some initial approximation.\footnote{Newton's method is indeed a special case of Hensel's lemma. See \cite{von1984hensel}.}

\begin{lemma}
    Let \(f \in \Integers_p[x]\) and \(f'\) be its derivative. If \(n\) and \(k\) are integers satisfying \(0 \leq 2k < n\), and for any \(\alpha \in \Integers_p\) we have \(f(\alpha) \equiv 0 \pmod{p^n}\) and \(\nu_p(f'(\alpha)) = k\), then there exists \(\beta \in \Integers_p\) such that \(f(\beta) \equiv 0 \pmod{p^{n + 1}}\), \(\nu_p(f'(\beta)) = k\), and \(\beta \equiv \alpha \pmod{p^{n-k}}\).
\end{lemma}

\begin{proof}
    We follow the proof in \cite{serre2012course}. Take \(\beta\) of the form \(\alpha + p^{n-k}\gamma\) with \(\gamma \in \Integers_p\). By Taylor's formula we have
    \[
        f(\beta) = f(\alpha) + p^{n-k}f'(\alpha)\gamma + p^{2n-2k}\zeta
    \]
    with \(\zeta \in \Integers_p\). Then by our hypothesis \(f(\alpha) = p^n\eta\) and \(f'(\alpha) = p^k\xi\) with \(\eta \in \Integers_p\) and \(\xi \in \Units\). This alows us to choose \(\gamma\) such that
    \[
        \eta + \gamma\xi \equiv 0 \pmod{p}.
    \]
    From this we have
    \[
        f(\beta) = p^n(\eta + \gamma\xi) + p^{2n-2k}\zeta \equiv 0 \pmod{p^{n+1}}
    \]
    since \(2n - 2k > n\). Finally, applying Taylor's formula to \(f'\) shows that \(f'(\beta) \equiv p^k \xi \pmod{p^{n-k}}\); since \(n - k > k\), we have \(\nu_p(f'(\beta)) = k\).
\end{proof}

\medskip

We can extend this result to polynomials in several variables.

\begin{theorem}
    Let \(f\) be a polynomial in \(m\) variables with coefficients in \(\Integers_p\) and let \(j\) be an integer such that \(0 \leq j \leq m\). If \(n\) and \(k\) are integers with \(0 \leq 2k < n\) and if \(f'(\alpha) \equiv 0 \pmod{p^n}\) and
    \[
        \nu_p\left(\frac{\partial f}{\partial a_j}(\alpha)\right) = k
    \]
    for some \(\alpha = (a_i) \in \Integers_p^m\), then there exists a zero \(\beta = (b_i) \in \Integers_p^m\) of \(f\) such that
    \[
        \alpha \equiv \beta \pmod{p^{n-k}}.
    \]
\end{theorem}

See \cite[pp.~14--15]{serre2012course} for a proof.

\medskip

By using Hensel's lemma, one can ``lift'' (i.e., obtain) from a given root \(r\) of a polynomial \(f\) modulo \(p^k\) another root \(s\) of \(f\) modulo \(p^{k+1}\) such that \(s \equiv r \pmod{p^k}\). The above theorem is often applied as the following corollary.

\begin{corollary}
    Every simple root of reduction modulo \(p\) of a polynomial \(f\) can be lifted to a root of \(f\) in \(\Integers_p[x]\).
\end{corollary}

Here a root \(x = (x_1, \dots, x_n)\) is considered ``simple'' if \(f(x) = 0\) but \(\partial f / \partial x_i \neq 0\) for some \(i\). This is a special case of the theorem with \(n = 1\) and \(k = 0\).

\smallskip

Let us now use Hensel's lemma to solve the congruence
\[
    x^2 \equiv 2 \pmod{7^3}.  
\]
The case \(x^2 \equiv 2 \pmod{7}\). This has solutions \(x \equiv 3\) and \(x \equiv 4 \pmod{7}\). Consider first the solution \(x \equiv 3 \pmod{7}\); then \(x = 3 + 7y\) for any integer \(y\). Substituting this into the congruence gives
\[
    9 + 42y + 49y^2 \equiv 0 \pmod{7^2},
\]
or equivalently,
\[
    7(1 + 6y) \equiv 1 + 6y \equiv 0 \pmod{7^2}.
\]
Thus \(y \equiv 1 \pmod{7}\) and we obtain \(x \equiv 10 \pmod{7^2}\). Similarly, for \(x \equiv 4 \pmod{7}\), we get \(x \equiv 39 \equiv -10 \pmod{7^2}\).

Now let us solve the initial congruence modulo \(7^3\). Consider again the case \(x \equiv 10 \pmod{7^2}\). Then \(x = 10 + 7^2z\) for some integer \(z\). Substituting this into the congruence gives
\[
    (10+7^2z)^2 \equiv 2 \pmod{7^3},
\]
and hence by routine computation we obtain \(z \equiv 2 \pmod{7}\) and thus \(x \equiv 10 + 7^2z \equiv 108 \pmod{7^3}\). Similarly, for \(x \equiv 39 \pmod{7^2}\), we get \(x \equiv 235 \pmod{7^3}\).

Thus we were able to ``lift'' solutions modulo \(7^2\) given solutions modulo \(7\). This process can be repeated indefinitely, and one can observe that the solutions modulo \(x^2 \equiv 2 \pmod{7^k}\) follow a familiar pattern:

\begin{align*}
    k = 1 & \qquad \pm 3 \\
    k = 2 & \qquad \pm (3 + 7) \\
    k = 3 & \qquad \pm (3 + 7 + 2 \times 7^2) \\
    k = 4 & \qquad \pm (3 + 7 + 2 \times 7^2 + 6 \times 7^3) \\
\end{align*}

\section{The Hilbert symbol.}\label{sec:hilbert-symbol}

We shall now quickly review the norm-residue symbol introduced by Hilbert in \cite[pp.~286--287]{hilbert1932theorie}. Our treatment shall be mainly cursory as our goal is to use the properties we shall develop here to simplify some of the arguments in our proof of the Hasse-Minkowski theorem. We refer the reader to \cite{serre2012course} for the details of the proofs.

\subsection{Definition and properties.} Throughout this section we take \(\field\) to be the field \(\Rationals_p\) or \(\Reals.\) For any \(a\) and \(b\) in \(\fieldU\) we define the Hilbert norm-residue symbol (or more simply, the Hilbert symbol) \((a, b)\) by
\[
    (a, b) = \begin{cases}
        \ \ 1 & \text{if } z^2 - ax^2 - by^2 = 0 \text{ has a nontrivial solution in } \field^3, \\
        \ \ -1 & \text{otherwise}.
    \end{cases}
\]
Because \((a, b)\) remains unchanged when \(a\) and \(b\) are multiplied by any non-zero square we see that the Hilbert symbol defines a map \(\SquareClass \times \SquareClass \to \{1, -1\}\). We now show some elementary properties of the Hilbert symbol.

\begin{theorem}
    The Hilbert symbol satisfies the following properties for all \(a, b, c, a' \in \fieldU\), (with \(a \neq 1\) whenever \(1-a\) appears in the formula):

    \smallskip

    \begin{enumerate}[nosep, label=(\alph*)]
        \item \((a, b) = (b, a)\);
        \item \((a, c^2) = 1\);
        \item \((a, -a) = -1\) and \((a, 1-a) = 1\);
        \item if \((a, b) = 1\), then \((aa', b) = (a', b)\); and
        \item \((a, b) = (a, -ab) = (a, (1-a)b)\).
    \end{enumerate}
\end{theorem}

See \cite[pp.~19--21]{serre2012course} for a proof.

\subsection{Hilbert reciprocity.}~Let \(\Places\) be the union of the set of all prime numbers and the symbol \(\infty\); we shall call \(\mathfrak{V}\) the set of ``places'' in \(\Rationals\). For each \(v \in \Places\), we write \((a, b)_v\) for  each \(a, b \in \fieldU\) to denote the image of \((a, b)\) under the map \(\SquareClass \times \SquareClass \to \{1, -1\}\) induced by the Hilbert symbol in \(\Rationals_v\), where we define \(\Rationals_{\infty}\) to be \(\Reals\).\label{sec:hilbert-reciprocity}

\begin{theoremx}
    If \(a, b \in \Rationals^{\times}\), then \((a, b)_v = 1\) for all but finitely many \(v \in \Places\) and 
    \[
        \prod_{v \in \Places} (a, b)_v = 1.
    \]
\end{theoremx}

See \cite[pp.~23--24]{serre2012course} for a proof. Note that this product formula is equivalent to the law of quadratic reciprocity.

\begin{theoremx}\label{thm:global-properties-hs}
    Let \(\{x_i\}_{i \in I}\) be a finite family of elements of \(\Rationals^{\times}\) and let \(\{\epsilon_{i, v}\}_{i \in I, v \in \Places}\) be a family of numbers equal to \(\pm 1\). There exists \(\lambda \in \fieldU\) such that \((x_i, \lambda)_v = \epsilon_{i, v}\) for all \(i \in I\) and \(v \in \Places\) if and only if the following conditions are satisfied:

    \smallskip

    \begin{enumerate}[nosep, label=(\alph*)]
        \item almost all the \(\epsilon_{i, v}\) are equal to \(1\);
        \item for all \(i \in I\), we have \(\prod_{v \in \Places} \epsilon_{i, v} = 1\); and
        \item for all \(v \in \Places\), there exists \(\lambda_v \in \Rationals_v^{\times}\) such that \((x_i, \lambda_v)_v = \epsilon_{i, v}\) for all \(i \in I\).
    \end{enumerate}
\end{theoremx}

We again refer the reader to \cite[pp.~24--26]{serre2012course} for a proof.

\section{Quadratic forms over \(\FiniteHead_q\) and \(\RationalsHead_p\).}

\subsection{Some results from Witt.}~In \S\,\ref{sec:mat-equiv} we have established that the forms \(xy\) and \(x^2 - y^2\) are equivalent over \(\Rationals\). As it turns out, these forms relate to more general forms over \(\Rationals\) and over \(\Rationals_p\) and are thus particularly useful in understanding quadratic forms over these fields, as has been demonstrated by the German mathematician Ernst Witt in his seminal paper from 1937 \cite{witt1937theorie}. We refer the reader to \cite{lam1973quadratic} for a more detailed treatment of Witt's decomposition and cancellation theorems, but we shall state here the results we need.\label{sec:results-from-witt}

We say that an element \(x\) of the quadratic space \((V,B)\) is \emph{isotropic} if it represents zero, i.e., if there exists a nonzero vector \(x \in V\) such that \(q(x) = 0\), where \(q\) is the associated quadratic form. We say that the whole space is isotropic if it contains an isotropic vector. If two isotropic elements \(x\) and \(y\) of \((V,B)\) satisfying \(B(x,y) \neq 0\) forms a basis of \(V\), then we call \((V, B)\) the hyperbolic plane. If \((V,B)\) is the hyperbolic plane then \(B\) is isomorphic to the orthogonal sum \(n\langle 1 \rangle \perp n \langle -1 \rangle\), where again \(\langle \lambda \rangle\) denotes the quadratic space of rank \(1\) with discriminant \(\lambda\), i.e., the isometry class of \(\lambda\). We can summarize these results in the following theorem.

\begin{theoremx}\label{thm:regular-witt}
    Let \((V, B)\) be a quadratic space over \(\Rationals\) of rank \(2\). Then the following statements are equivalent:

    \smallskip

    \begin{enumerate}[nosep, label=(\alph*)]
        \item \((V, B)\) is regular and isotropic.
        \item \(B \cong \langle 1, -1 \rangle \cong \langle 1 \rangle \perp \langle -1 \rangle\).
        \item \(B\) is regular and \(\discr B \in -1 \cdot \SquareClass\).
        \item \(B\) is isomorphic to the equivalence class of the form \(xy\).
    \end{enumerate}
\end{theoremx}

\emph{Proof.} The results are immediate but cf. \cite[pp.~12--13]{lam1973quadratic}.

\begin{theoremx}\label{thm:hyperbolic-decomp}
    Let \((V,B)\) be a regular quadratic space and let \(x \neq 0\) be an isotropic element of \(V\). Then there exists a subspace \(W\) of \(V\) such that \(W\) contains \(x\) and is hyperbolic.
\end{theoremx}

\begin{proof}
    See also \cite[p.~13]{clarkquadratic} or \cite[p.~29]{serre2012course}. Since \((V,B)\) is regular, there exists a vector \(z \in V\) such that \(B(x,z) \neq 0\); we can assume without loss of generality that \(B(x,z) = 1\). We claim that there exists a \(\lambda \in \field\) such that \(q(\lambda x + z) = 0\) where \(q\) is the associated quadratic form. Indeed we have
    \[
        q(\lambda x + z) = \lambda^2q(x) + 2\lambda B(x,z) + q(z) = 2\lambda + q(z).     
    \]
    Let \(\lambda = -q(z)/2\). Then if \(y = \lambda x + z\), we have \(q(x) = q(y) = 0\) and
    \[
        B(x,y) = B(x, \lambda x + z) = \lambda q(x) + B(x,z) = 1.
    \]
    Thus the subspace spanned by \(x\) and \(y\) is hyperbolic.
\end{proof}

\medskip


We say that a quadratic space \((V,B)\) is \emph{universal} if it represents every element of \(\fieldU\), i.e., \(D(B) = \fieldU\).  We can deduce the following corollary.

\begin{corollaryx}\label{cor:regular-isotropic}
    Every regular isotropic quadratic space is universal.
\end{corollaryx}

The following corollary can also be deduced from the above theorem.

\begin{corollaryx}[First representation theorem]\label{cor:rep-theorem-1}
    Let \(q\) be a nondegenerate quadratic form and let \(\lambda \in \fieldU\). Then \(\lambda \in D(q)\) if and only if \(q \perp \langle -\lambda \rangle\) is isotropic.
\end{corollaryx}

\begin{proof}
    See \cite[pp.~14--15]{lam1973quadratic}. We can assume without loss of generality that \(q\) is a diagonal form. Now suppose \(\lambda \in D(q)\). Then \(\lambda = \sum \lambda_i x_i^2\) for some \(x_i \in \field\), so that \(\left(\sum \lambda_ix_i^2 + (-\lambda) \cdot 1^2 \right) = 0\) and hence \(q \perp \langle -\lambda \rangle\) is isotropic. Conversely, if \(q \perp \langle -\lambda \rangle\) is isotropic, then there exists a vector \(x = (x_1, \dots, x_{n+1})\) such that \(q(x) = 0\). If \(x_{n+1} \neq 0\), then
    \[
        \lambda = \sum \lambda_i\left(\frac{x_i}{x_{n+1}}\right)^2 \in D(q).
    \]
    Otherwise, if \(x_{n+1} = 0\), then \(x' = (x_1, \dots, x_n) \neq 0\) is an isotropic vector of \(q\), whence \(D(q) = \fieldU\) and \(\lambda \in D(q)\).
\end{proof}

We now ``translate'' Theorem \ref{thm:hyperbolic-decomp} and its corollaries in the language of quadratic forms. To simplify things, we shall again abuse notation and write \(q + r\) for the orthogonal sum of quadratic forms \(q\) and \(r\) of rank \(n\) and \(m\) respectively, i.e.,
\[q + r = f(x_1, \dots, x_n) + g(x_{n+1}, \dots, x_{n+m}).\]
Similarly, we put \(q - r\) for \(q + (-r)\). Moreover, we say that \(q\) is hyperbolic if
\[
    q \sim xy \sim x^2 - y^2.
\]
The following theorem is therefore a restatement of Theorem \ref{thm:hyperbolic-decomp} and Corollary \ref{cor:regular-isotropic} above:

\begin{theoremx}\label{thm:hyperbolic-decomp-2}
    Let \(q\) be a nondegenerate quadratic form representing zero. Then \(q\) is equivalent to the form \(q' + r\) where \(q'\) is hyperbolic and \(r\) is of rank \(n - 2\). Moreover, \(q\) is universal.
\end{theoremx}

\begin{corollary}
    Let \(q\) be a nondegenerate quadratic form in \(n - 1\) variables and let \(\lambda \in \fieldU\). Then the following statements are equivalent:

    \smallskip

    \begin{enumerate}[nosep, label=(\alph*)]
        \item The form \(q\) represents \(\lambda\).
        \item If \(r\) is a quadratic form in \(n - 2\) variables, then \(q \sim r + \lambda z^2\) for some variable \(z\).
        \item The form \(s = q - \lambda z^2\) represents zero.
    \end{enumerate}
\end{corollary}

\begin{proof}
    The equivalence (a) \(\iff\) (c) is Corollary \ref{cor:rep-theorem-1} (the first representation theorem) of Theorem \ref{thm:hyperbolic-decomp} above. The implication (b) \(\implies\) (a) is trivial. For the converse, suppose \(q\) represents \(\lambda\); then there exists a vector \(x\) such that \(q(x) = \lambda\). Let \(H\) be the orthogonal complement of \(x\) so that \(V = H \perp \langle \lambda \rangle.\) If \(r\) is the quadratic form associated with \(H\), then \(q \sim r + \lambda z^2\) for some variable \(z\).
\end{proof}

\subsection{Quadratic forms over \(\FiniteField_q\).}~We first look at quadratic forms over a finite field \(\FiniteField_q\) with \(q\) elements where \(q\) is a power of an odd prime \(p\). The two theorems in this section show us that quadratic forms over \(\FiniteField_q\) are completely determined, up to equivalence, by their rank and discriminant. The results in this section are not needed in the sequel.\label{sec:quadratic-forms-fq}

\begin{theoremx}{\normalfont\cite[p.~34]{serre2012course}}
    A quadratic form over \(\FiniteField_q\) of rank \(r \geq 2\) (resp. \(r \geq 3\)) is universal in \(\FiniteField_q^\times\) (resp. \(\FiniteField_q\)).\label{thm:universal-forms-in-fq}
\end{theoremx}

\begin{proof}
    Following the corollary of Theorem \ref{thm:hyperbolic-decomp-2} of the preceding section, it suffices to show that all quadratic forms in \(3\) variables represent \(0\). This fact can indeed be shown as true, and is a consequence of the Chevalley-Warning theorem, which states that in a finite field if the number of variables of a polynomial is greater than its degree, then the cardinality of its solution set (i.e., the number of zeroes it has) is divisible by the characteristic of the field. We have not taken up this result in this paper but the reader may refer to \cite[p.~5]{serre2012course} for a proof.
\end{proof}

\begin{theoremx}\label{thm:quadratic-forms-fq-rank-n}
    Every nondegenerate form over \(\FiniteField_q\) of rank \(n\) is equivalent to the form
    \[
        x_1^2 + \cdots + x_{n-1}^2 + x_n^2.
    \]
    or to the form
    \[
        x_1^2 + \cdots + x_{n-1}^2 + \lambda x_n^2
    \]
    where \(\lambda\) is a nonsquare element of \(\FiniteField_q\), depending on whether its discriminant is a square or not.
\end{theoremx}

\begin{proof}
    We proceed by induction. Let \(q\) be a quadratic form of rank \(n\) in \(\FiniteField_q\). For \(n = 1\), every quadratic form is of the form \(x^2\) or \(ax^2\) since the group \(\FiniteField_q^{\times} / \FiniteField_q^{\times2}\) has order \(2\). If \(n \geq 2\) then by Theorem \ref{thm:universal-forms-in-fq} \(q\), being universal, represents \(1\); thus \(q \sim x^2 + r\) for some quadratic form \(r\) of rank \(n - 1\) and by induction, we are done.
\end{proof}

\subsection{The Hasse-Minkowski invariant.}~Let \(\field\) be the field of \(p\)-adic numbers of \(\Reals\) (we shall maintain this assumption until the end of the next section). Let \((V, B)\) be a quadratic space over \(\field\) with \(q\) as its associated quadratic form and let \(\discr q\) be its discriminant. We have established in \S\,\ref{sec:square-classes} that \(\discr B\) is an element of the square class \(\SquareClass\). Now if \(\Basis = \{b_1, \dots, b_n\}\) is an orthogonal basis of \(V\) and we write \(a_{i} = B(b_i, b_i)\), then we have\label{sec:hasse-invariant}
\[
    \discr q = a_1 \cdots a_n.  
\]
Recall now that we have defined the Hilbert symbol \((a, b)\) for elements \(a, b \in \fieldU\). We then define
\[
    \epsilon(\Basis) = \prod_{i < j} (a_i, a_j)
\]
so that \(\epsilon := \epsilon(\Basis) = \pm 1\). We now have the following theorem.

\begin{theorem}{\normalfont\cite[pp.~35--36]{serre2012course}}
    The number \(\epsilon\) does not depend on the choice of the orthogonal basis \(\Basis\) of \(V\).
\end{theorem}

\begin{proof}
    We prove the theorem by induction on the rank of \(V\). If \(n = 1\), we have \(\epsilon = 1\) as the product is trivial. If \(n = 2\), then \(\epsilon = 1\) if and only if \(z^2 - ax^2 - by^2\) has a nontrivial solution in \(\field^3\), which is true, by the corollary of Theorem \ref{thm:hyperbolic-decomp-2} of the preceding section, if and only if \(ax^2 + by^2\) represents \(1\). But this last statement implies that there exists a vector \(v \in V\) such that \(q(x) = 1\); this vector exists independent of the choice of basis.

    For \(n \geq 3\), we proceed by induction on \(n\). By the theorem of \S\,\ref{sec:contiguous-bases}, it suffices to show that \(\epsilon(\Basis) = \epsilon(\Basis')\) whenever \(\Basis\) and \(\Basis'\) are contiguous bases of \(V\). Moreover, because the Hilbert symbol is symmetric, \(\epsilon(\Basis)\) does not change no matter how we permute the elements of \(\Basis\). Thus we can assume without loss of generality that \(b_1 = b'_1\) and hence \(a_1 = a'_1\). We then have
    \begin{align*}
        \epsilon(\Basis)  &= (a_1, a_2 \cdots a_n) \prod_{2 \leq i <j} (a_i, a_j) \\
                                &= (a_1, \discr B a_1) \prod_{2 \leq i <j} (a_i, a_j)
    \end{align*}
    since \(\discr B = a_1 \cdots a_n\). Similarly, we have
    \[
        \epsilon(\mathfrak{B'}) = (a_1, \discr B a_1) \prod_{2 \leq i <j} (a'_i, a'_j).
    \]
    But applying the inductive hypothesis to the orthogonal complement of \(b_1\) gives us
    \[
        \prod_{2 \leq i <j} (a_i, a_j) = \prod_{2 \leq i <j} (a'_i, a'_j),  
    \]
    which completes the proof.
\end{proof}

In view of the above theorem we shall call \(\epsilon\) the \emph{Hasse-Minkowski invariant}\index{Hasse-Minkowski invariant} of the quadratic space \((V, B)\). We shall also sometimes write \(\epsilon(B)\) or \(\epsilon(q)\) (where \(q\) is the form associated with \(B\)) to emphasize the dependence of \(\epsilon\) on \(B\) or \(q\).\footnote{The notation \(c_p(q)\) for the Hasse-Minkowski invariant of the form \(q\) also appears in the literature, as in, e.g., \cite{jones1950arithmetic,cassels2008rational}.}

\subsection{Representation in \(\Rationals_p\).}~In this section we provide the necessary and sufficient conditions for any form over \(\Rationals_p\) to represent zero.\label{sec:representation-in-qp-sec}

\begin{theoremx}\label{thm:rep-in-rationals-p}
    Let \(q\) be a nondegenerate quadratic form over \(\Rationals_p\) of rank \(n\) and let \(\discr q\) and \(\epsilon\) be its discriminant and Hasse-Minkowski invariant respectively. Then for \(q\) to represent \(0\) in \(\Rationals_p\) it is necessary and sufficient that:

    \smallskip

    \begin{enumerate}[nosep, label=(\alph*)]
        \item \(n = 2\) and \(\discr q = -1\) (in \(\SquareClass\));
        \item \(n = 3\) and \((-1, \discr q) = \epsilon\);
        \item \(n = 4\), \(\discr q = \pm 1\) and \(\epsilon = (-1, -1)\); or
        \item \(n \geq 5\).
    \end{enumerate}
    That is, all forms of rank \(n \geq 5\) represent \(0\) in \(\Rationals_p\).
\end{theoremx}

We shall not prove this theorem in this paper but the reader may refer to \cite[pp.~36--39]{serre2012course} for a proof. Now note that if \(\lambda \in \SquareClass\) and \(q_{\lambda}\) is the form \(q - \lambda z^2\), then we know by the results in \S\,\ref{sec:results-from-witt} that \(q_{\lambda}\) represents \(0\) in \(\Rationals_p\) if and only if \(q\) represents \(\lambda\). Noting also the fact that \(\discr q_{\lambda} = -\lambda \discr q\) (cf. Theorem \ref{thm:regular-witt} of \S\,\ref{sec:results-from-witt}) and \(\epsilon(q_\lambda) = (-\lambda, \discr q)\epsilon\), we can deduce the following corollary, which provides necessary and sufficient conditions for a form \(q\) to represent any \(\lambda \in \SquareClass\), not just \(0\).

\begin{corollary}
    Let \(\lambda \in \SquareClass\) and let \(q\), \(\discr q\) and \(\epsilon\) be as above. Then for \(q\) to represent \(\lambda\) in \(\Rationals_p\) it is necessary and sufficient that:

    \smallskip

    \begin{enumerate}[nosep, label=(\alph*)]
        \item \(n = 1\) and \(\lambda = \discr q\);
        \item \(n = 2\) and \((\lambda, -\discr q) = \epsilon\);
        \item \(n = 3\) and \(\lambda = \pm \discr q\) and \(\epsilon = (-1, \discr q)\); or
        \item \(n \geq 4\).
    \end{enumerate}
\end{corollary}

With these results, we are now able to classify, up to equivalence, all quadratic forms over \(\Rationals_p\). We summarize the results in the following theorem.

\begin{theoremx}{\normalfont\cite[p.~39]{serre2012course}}
    Two quadratic forms over \(\field = \Rationals_p\) or \(\Reals\) are equivalent if and only if they have the same rank, discriminant and Hasse-Minkowski invariant.
\end{theoremx}

\begin{proof}
    The forward implication follows directly from the definition of equivalence. We shall prove the converse by induction on the rank \(n\) of two  forms \(q\) and \(r\). We also write \(\Delta\) and \(\epsilon\) for the discriminant \(\Delta q = \Delta r\) and the Hasse-Minkowski invariant \(\epsilon(q) = \epsilon(r)\). The case \(n = 0\) is trivial. By the corollary of Theorem\,\ref{thm:rep-in-rationals-p} of this section, \(q\) and \(r\) represent the same element \(\lambda \in \fieldU\). We then have
    \[
        q \sim \langle \lambda \rangle \perp q' \quad \text{and} \quad r \sim \langle \lambda \rangle \perp r'
    \]
    where \(q'\) and \(r'\) are quadratic forms of rank \(n - 1\). We then have
    \[
      \discr q' = \lambda \discr q = \lambda \discr r = \discr r'
    \]
    and 
    \[
        \epsilon(q') = (-\lambda, \discr q')\epsilon = (-\lambda, \discr r')\epsilon = \epsilon(r'),
    \]
    and thus \(q'\) and \(r'\) have the same rank, discriminant and Hasse-Minkowski invariant. By the inductive hypothesis, \(q'\) and \(r'\) are equivalent, and hence \(q\) and \(r\) are equivalent.
\end{proof}

We conclude this section by exploring in a bit more detail quadratic forms over the field \(\Reals\) of real numbers. These results are again from \cite{serre2012course}, but shall not be needed in the sequel. We know from \S\,\ref{sec:results-from-witt} that if \(q\) is a quadratic form over \(\Reals\) of rank \(n\), then \(q\) is equivalent to the form
\[
    x_1^2 + \cdots + x_{r}^2 - y_1^2 - \cdots - y_{s}^2
\]
where \(r\) and \(s\) are positive integers satisfying \(r + s = n\). We shall now show that the integers \(r\) and \(s\) depend only on the form \(q\) and is thus an invariant of \(q\) over \(\Reals\). We shall call the pair \((r,s)\) the \emph{signature}\index{signature} of \(q\). We say that \(q\) is \emph{positive definite} if \(r = n\) and \emph{negative definite} if \(s = n\). We say that \(q\) is \emph{indefinite} if \(r\) and \(s\) are both positive. If we let \(\epsilon\) and \(\discr q\) be the Hasse-Minkowski invariant and discriminant of \(q\) respectively, then, because \((-1, -1) = -1\),
\[
    \epsilon = (-1)^{s(s-1)/2} \quad \text{and} \quad \discr q = (-1)^{s}.
\]
This result is Sylvester's law of inertia from linear algebra. See also \cite[pp.~83--84]{szymiczek2017bilinear}.

\section{The Hasse-Minkowski theorem.}

\subsection{The Hasse-Minkowski theorem.}~We are now ready to prove the main result of this chapter, the Hasse-Minkowski theorem. As in \S\,\ref{sec:hilbert-reciprocity}, we shall write \(\mathfrak{V}\) for the union of the set of prime numbers and the symbol \(\infty\), so that writing \(\Rationals_v\) for the \(v\)-adic field is well-defined for all \(v \in \mathfrak{V}\), with \(\Rationals_{\infty} = \Reals\) by definition. As before, we shall write \(\quadform{a_1, \dots, a_n}\) for the quadratic form \(q \sim a_1x_1^2 + \cdots + a_nx_n^2\) over \(\Rationals\); we then associate with \(q\):\label{sec:hasse-minkowski} 

\smallskip

\begin{enumerate}[nosep, label=(\alph*)]
    \item a discriminant \(\Delta q\) in \(\Rationals^{\times} / \Rationals^{\times 2}\) equal to \(a_1 \cdots a_n\);
    \item for any \(v \in \Places\) we define the injection \(q_v : \Rationals \hookrightarrow \Rationals_v\) so that \(q_v\) denotes the quadratic form \(q\) over \(\Rationals_v\), such that \(\discr_v q\) is the image of \(\discr q\) by \(\Rationals^{\times} / \Rationals^{\times 2} \to \Rationals_v^{\times} / \Rationals_v^{\times2}\) and
    \[
      \epsilon_v (q) = \prod_{i < j} (a_i, a_j)_v,  
    \]
    with \(\prod \epsilon_v(q) = 1\) over all the places \(v\); and
    \item a signature \((r,s)\) as defined in the preceding section.
\end{enumerate}

The numbers \(\discr_v q\), \(\epsilon_v(q)\) and \((r, s)\) all called the \emph{local invariants} of \(q\) with respect to the place \(v\). With our notation fixed, we can now state the Hasse-Minkowski theorem.

\begin{theoremx}\label{thm:hasse-minkowski}
    In order that a quadratic form over \(\Rationals\) represent zero, it is necessary and sufficient that for every place \(v\) of \(\Rationals\) the form \(q_v\) represents \(0\).
\end{theoremx}

Before proceeding with the proof, recall that in \S\,\ref{sec:polynomials-over-zp} we have remarked that, while the polynomial \(x^2 - 2\) has a solution in \(\Rationals_7\), we know for a fact that it does not have a solution in \(\Rationals\). We see that this is the case because, as the Hasse-Minkowski theorem tells us, a quadratic form has a solution in \(\Rationals\) if and only if it has a solution in \(\Rationals_v\) for all \(v \in \Places\). Indeed we see by routine verification that \(x^2 - 2\) has no solutions in \(\Rationals_3\).


\emph{Proof.} The necessity is clear since \(\Rationals \subset \Rationals_p\) for all \(p\) so we are going to prove sufficiency. Write the form \(q\) as its equivalent diagonal form \(\quadform{a_1, \dots, a_n}\) where each \(a_i \in \fieldU\). Replacing \(q\) with \(a_1 q\), we can assume, without loss of generality, that \(a_1 = 1\). We shall then again follow \cite[pp.~41--44]{serre2012course} and \cite[pp.~102--104]{gerstein2008basic} and prove the theorem for \(n = 2, 3, 4\) and \(n \geq 5.\)

\begin{enumerate}[wide, nosep, label=(\alph*)]
    \item \emph{Case 1, \(n = 2\).} Suppose \(q \cong \quadform{1, a}\). Then for \(q\) to represent \(0\) in \(\Rationals\), \(q\) must represent \(0\) in all the places \(v\) of \(\Rationals\); in particular, \(q\) must represent \(0\) in \(\Rationals_{\infty} = \Reals\). It therefore follows that \(a < 0\) and thus we only need to consider the form \(q \cong \quadform{1, -a}\), with \(0 < a \in \Rationals\), so that \(q\) represents \(0\) in \(\Rationals\) if and only if it is a square in \(\Rationals\). Since \(q\) represents \(0\) in all \(\Rationals_p\),  \(a\) is a square in \(\Rationals_p\). Thus writing \(a\) as the product \(\prod_p p^{\nu_p(a)}\), we have that \(\nu_p(a)\) is even and therefore, \(a\) is a square in \(\Rationals\) and hence \(q\) represents \(0\) in \(\Rationals\).
    
    \item \emph{Case 2, \(n = 3\).} We can assume, without loss of generality, that \(q \cong \quadform{1, -a, -b}\) where \(a\) and \(b\) are squarefree integers (i.e. \(\nu_p(a)\) and \(\nu_p(b)\) are equal to \(0\) or \(1\) for all prime \(p\)) and \(|a| \leq |b|\). We shall prove this by induction on the integer \(m = |a| + |b|\). If \(m = 2\), then \(q \cong \quadform{1, \pm 1, \pm 1}\), but since \(q\) is represents 0 in \(\Rationals_\infty = \Reals\), then \(q \not\cong \quadform{1, 1, 1}\) and hence \(q \cong \quadform{1, -1, -1}\).
    
    Now if \(m > 2\). Recall that we are assuming \(|a| \leq |b|\) so that \(|b| \geq 2\) and write \(b = \pm p_1 \cdots p_r\) where \(p_i\) are distinct primes (noting also our assumption that \(b\) is squarefree). We then use the following lemma:

    \begin{lemma}
        Let \(a\) and \(b\) be as above and let \(p\) be a prime number. Then {\normalfont (i)} \(a\) is a square modulo \(p\); and {\normalfont (ii)} \(a\) is a square modulo \(b\).
    \end{lemma}

    We shall not prove this lemma but see \cite[p.~42]{serre2012course} for a proof. Here (ii) is a corollary of (i). Now since \(a\) is a square modulo \(b\), there exists integers \(t\) and \(b'\) such that \(t^2 = a + b'b\) such that \(0 \leq |t| \leq |b|/2\). We then have
    \[ b'b = N(t + \sqrt{a})\]
    where \(N: \field(\sqrt{a})/\field \to \field\) is the norm mapping of the extension \(\field(\sqrt{a})/\field\) of \(\field\) where \(k\) is \(\Rationals\) or \(\Rationals_v\) (cf. \cite[p.940]{rotman2010advanced}, \cite[p.~289]{hungerford2012algebra}). We can then conclude that \(q\) represents \(0\) in \(\field\) if and only if the form \(q' \cong \quadform{1, -a, -b'}\) does. And since \(|b| \geq 2\), it follows that
    \[
        |b'| = \left|\frac{t^2-a}{b}\right| \leq \frac{|b|}{4} + 1 \leq |b|
    \]
    Again write \(b'\) as \(b''u^2\) for some squarefree integer \(b''\) and some integer \(u\). It follows \emph{a fortiori} that \(|b''| < |b|\) and thus the form \(q'' \cong \quadform{1, -a, -b''}\) represents \(0\) in \(\Rationals\) by the inductive hypothesis, whence \(q\) represents \(0\) in \(\Rationals\), as desired.

    \item \emph{Case 3, \(n = 4\).} Let \(q \cong \quadform{a, b, -c, -d}\). By Theorem\,\ref{thm:hyperbolic-decomp-2}, it suffices to show that there exists some \(\lambda \in \fieldU\) (where again \(\field\) is either \(\Rationals\) or \(\Rationals_v\)) represented by both \(\quadform{a, b}\) and \(\quadform{-c, -d}\). Suppose \(\quadform{a, b, -c, -d}\) represents \(0\) in each \(\Rationals_v\); then \(\quadform{a, b}\) and \(\quadform{c, d}\) represent the same element, say \(\lambda_v\), in \(\Rationals_v\). Over \(\Rationals_v\), it follows that \(\quadform{a, b}\) represents \(\lambda_v\) if and only if \(\quadform{a, b, -\lambda_v}\) represents \(0\) if and only if \(\quadform{1, ab, -a\lambda_v}\) is isotropic. This is true if and only if \((-ab, a\lambda_v)_v = 1\), which is true if and only if \((a, b)_v = (-ab, \lambda_v)_v = 1\). Similarly, we have \((c,d)_v = (-cd, \lambda_v)_v = 1\).
    
    Using the product formula of \S\,\ref{sec:hilbert-reciprocity}, we have 
    \[
        \prod_{v \in \Places} (a, b)_v = \prod_{v \in \Places} (c, d)_v = 1,
    \]
    and thus by Theorem\,\ref{thm:global-properties-hs} of \S\,\ref{sec:hilbert-reciprocity} we can find a \(\lambda \in \Rationals^{\times}\) satisfying
    \[
        (a, b)_v = (-ab, \lambda)_v \quad \text{and} \quad (c, d)_v = (-cd, \lambda)_v,
    \]
    and thus the forms \(\quadform{a, b, \lambda}\) and \(\quadform{c, d, \lambda}\) are isotropic in each \(\Rationals_v\). By our proof in the case \(n = 3\) above, it follows that \(\quadform{a, b, \lambda}\) and \(\quadform{c, d, \lambda}\) are isotropic in \(\Rationals\) and thus both \(\quadform{a,b}\) and \(\quadform{c,d}\) represent \(\lambda\) in \(\Rationals\). Thus \(\quadform{a, b, -c, -d}\) represents \(0\) in \(\Rationals\) and the result follows.
    
    \item \emph{Case 4, \(n \geq 5\).} Let \(q = \quadform{a_1, \dots, a_n}\) be a quadratic form of rank \(n \geq 5\). Let \(r = \quadform{a_1, a_2}\) and \(s = \quadform{-a_3, \dots, -a_n}\), so that \(q \cong r \perp s\), i.e., \(q = r + s\) by the notation we introduced in \S\,\ref{sec:results-from-witt} 
    ADD QED
\end{enumerate}

\medskip

We can deduce the following corollary, which is a straightforward application of the above theorem to the form \(\quadform{\lambda} - q\).

\begin{corollary}
    % \label{cor:hasse-minkowski-representation}
    For a quadratic form \(q\) to represent \(\lambda \in \Rationals^{\times}\), it is necessary and sufficient that the form \(q_v\) represents \(\lambda\) in \(\Rationals_v\) for all \(v \in \Places\).
\end{corollary}

The next theorem relates the Hasse-Minkowski theorem with our notion of equivalence.\footnote{One may sometimes find Theorem\,\ref{thm:hasse-minkowski-equivalence}, and not Theorem\,\ref{thm:hasse-minkowski}, referred to as the Hasse-Minkowski theorem in the literature.}

\begin{theoremx}\label{thm:hasse-minkowski-equivalence}
    Two quadratic forms over \(\Rationals\) are equivalent if and only if they are equivalent over \(\Rationals_v\) for all \(v \in \Places\).
\end{theoremx}

\begin{proof}
    Let \(q\) and \(r\) be equivalent quadratic forms over \(\Rationals\). Necessity follows directly from the definitions. We prove sufficiency by induction on the rank \(n\) of \(q\) and \(r\). The case \(n = 0\) is trivial. We thus suppose \(n > 0\), and by the corollary of Theorem\,\ref{thm:hasse-minkowski} above, we can assume that \(q\) and \(r\) represent the same element \(\lambda \in \Rationals^{\times}\). By the representation criterion of \S\,\ref{sec:diagonalization-of-forms}, \(q \sim \quadform{\lambda} \perp q'\) and \(r \sim \quadform{\lambda} \perp r'\) where \(q'\) and \(r'\) are quadratic forms of rank \(n - 1\) over \(\Rationals\). This implies \(r \sim r'\) over \(\Rationals_v\) since \(q \sim q'\) and \(\quadform{\lambda} \sim \quadform{\lambda}\) for every \(\Rationals_v\). From our induction hypothesis \(r \sim r'\) over \(\Rationals\) and hence, similarly, \(q \sim q'\) over \(\Rationals\).
\end{proof}

% \subsection{Sum of squares.}~We now begin to move from the rational theory of quadratic forms towards an initial result involving integral results. We shall discuss integral results in more detail in the next chapter. We now use the results of this chapter to prove some classical results, due to Gauss and Lagrange, on the representation of integers as sums of squares. We follow the treatment in \cite[pp.~45--47]{serre2012course} and \cite[pp.~109--111]{gerstein2008basic}. We begin with a couple of lemmas.

% \begin{lemmax}
%     If \(\lambda \in \Rationals^{\times}\), then \(\lambda\) is a sum of three rational squares if and only if \(\lambda > 0\) and \(-\lambda\) is not a square in \(\Rationals_2\).
% \end{lemmax}

% \begin{lemmax}
    
% \end{lemmax}

% We now arrive to the main result of this section, which is due to Gauss.
% \begin{theorem}
%     A positive integer \(n\) is a sum of three squares if and only if \(n\) is not of the form \(4^a k\) for some integer \(a\), where \(k \equiv 7 \pmod{8}\).
% \end{theorem}

% We can deduce the following corollary, due to Lagrange and commonly known as the four-square theorem, as mentioned in the Preface.

% \begin{corollary}
%     Every positive integer is a sum of four squares.
% \end{corollary}

