\chapter{A local-global principle}
\label{chap:local-global-principle}

{\scshape In this} chapter we shall prove the Hasse-Minkowski theorem for
quadratic forms over \(\Rationals\). We shall follow Chapters 2\,--\,4 of
\cite{serre2012course}, introducing and developing the structure theorems for
the ring \(\Integers_p\) of \(p\)-adic integers and the field \(\Rationals_p\)
of \(p\)-adic numbers together with the main result from Hensel on the roots of
polynomials over \(\Integers_p\), followed by the introduction of the Hilbert
symbol and some of its properties, and culminating in the proof of the
Hasse-Minkowski theorem.

\section{The field of \(p\)-adic numbers}

\subsection{}~The history of the theory of quadratic forms over \(\Rationals\)
owes much of its development to the theory of \(p\)-adic numbers as introduced
by Hensel and expanded by his student
Hasse.\,\cite{hasse1922uber,hensel1913zahlentheorie} In this section, we shall
introduce the field of \(p\)-adic numbers and prove some of its basic properties
and develop the results we need as they relate to quadratic forms. For a more
detailed treatment of the theory of \(p\)-adic numbers, we refer the reader to
\cite{gouvea1997p,koblitzp}. The \(p\)-adic field can be constructed both
analytically and algebraically; our approach in this paper shall be mainly
algebraic.\label{sec:field-of-p-adic-numbers}

% The theory of \(p\)-adic numbers has been motivated historically by the desire
% to find the solutions to congruences of the form \begin{equation} x^2 \equiv a
% \pmod{p^n} \label{eq:congruence-mod-pn} \end{equation} for a prime number
% \(p\) and an integer \(a\) and for all \(n \geq 1\). \cite{amice1975nombres}
% Consider, for example, the case \(p = 5\) and \(a = -1\). For \(n = 1\), this
% can be simplified to \(x^2 \equiv -1 \pmod{5}\), which has the solutions \(x
% \equiv 2\) and \(x \equiv -2 \pmod{5}\). Observe that for any \(k \leq n\)

% \subsection{}~
We begin with a formal definition, following \cite{serre2012course}. Let \(A_n
:= \Integers/p^n \Integers\) be the set of equivalence classes of integers
modulo \(p^n\) for a prime number \(p\) for each \(n \in \Naturals\). Each
\(A_n\) is a ring and induces a homomorphism\label{sec:padic-def}
\[
  \phi_n : A_{n} \to A_{n - 1}  
\]
which is surjective and has kernel \(p^{n-1}A_n\). Define the sequence of rings
and the corresponding homomorphisms
\begin{equation*}
    \dots \to A_n \xrightarrow{\phi_n} A_{n - 1} \xrightarrow{\phi_{n - 1}} \dots \xrightarrow{\phi_2} A_1
\end{equation*}
indexed by \(\Naturals\). The \emph{ring of \(p\)-adic integers} \(\Integers_p\)
is defined to be the inverse limit of this sequence, that is,
\[
  \Integers_p := \varprojlim A_n = \left\{
    (a_n)_{n \in \Naturals} \in \prod_{n \in \Naturals} A_n : \phi_n(a_n) = a_{n - 1} \text{ for all } n \in \Naturals
  \right\}.
\]
We shall see shortly that this is indeed a ring. Each element of \(\Integers_p\)
is a sequence of the form \((\dots, x_3, x_2, x_1)\) and thus it is often useful
to think of an element \(\alpha\) of \(\Integers_p\) as a formal sum
\[
  \alpha = a_0 + a_1p + a_2p^2 + \cdots
\]
Addition and multiplication in \(\Integers_p\) are defined componentwise, that
is, if \(\alpha = (a_n)_{n \in \Naturals}\) and \(\beta = (b_n)_{n \in
\Naturals}\) are elements of \(\Integers_p\), then \(\alpha + \beta = (a_n +
b_n)_{n \in \Naturals}\) and \(\alpha\beta = (a_nb_n)_{n \in \Naturals}\).
Elements of \(\Integers_p\) are called \emph{\(p\)-adic integers}.

We proceed by exploring some properties of \(\Integers_p\). First fix a natural
number \(n\) and define the projection \(\pi_n: \Integers_p \to A_n\) which
associates with every \(p\)-adic integer \(\alpha\) its \(n\)-th component
\(a_n\). Let \([p^n]\) be the map from \(A_n\) to itself defined by
multiplication by \(p^n\). Then we have
\begin{theoremx}\label{thm:exact-sequence-zp} The sequence \(0 \to \Integers_p
    \xrightarrow{[p^n]} \Integers_p \xrightarrow{\pi_n} A_n \to 0\) is a short
    exact sequence of abelian groups.
\end{theoremx}

See \cite[pp.~11--12]{serre2012course} for a proof.
Theorem~\ref{thm:exact-sequence-zp} has the following immediate consequence:

\begin{corollary}
    For all positive integers \(m\), \(\Integers_p / p^m \Integers_p \cong
    \Integers / p^m \Integers\).
\end{corollary}

\begin{theoremx}\label{thm:criterion-units-of-zp} An element of \(\Integers_p\)
    is invertible if and only if it is not divisible by \(p\).
\end{theoremx}

\begin{proof}
    Let \(\alpha = (a_n) \in \Integers_p.\) If \(p\) does not divide \(\alpha\)
    then \(p\) does not divide each of the \(a_n\) and in particular \(a_1
    \not\equiv 0 \pmod{p}\). By the Euclidean algorithm, there exists integers
    \(\xi\) and \(\eta\) such that \(a_1\xi + p\eta = 1\). Because \(p \mid
    p\eta\), it follows that \(a_1\xi \equiv 1 \pmod{p}\). Let \(z = 1 -
    \xi\alpha\); we can then compute the inverse of \(\alpha\) as follows:
\end{proof}

\begin{theoremx}\label{thm:units-of-zp} Let \(\Units\) be the set of invertible
    elements of \(\Integers_p\). Then every element of \(\Integers_p\) can be
    written uniquely as \(\alpha = p^nu\) where \(n \geq 0\) and \(u \in
    \Units\).
\end{theoremx}

\begin{proof}
    If \(\alpha \in \Integers_p\) is not zero, then there exists a largest
    integer \(n\) such that \(a_n = \pi_n(a) = 0\); then \(\alpha = p^nu\) where
    \(u\) is not divisible by \(p\) (and hence by Theorem
    \ref{thm:criterion-units-of-zp} is invertible). That \(n\) is unique follows
    from the fact that \(p^n\) is injective.
\end{proof}

\smallskip

The unique integer \(n\) defined in Theorem \ref{thm:units-of-zp} induces a
function \(\nu_p : \Integers_p \to \Integers_{\geq 0}\), i.e., we shall write
\(\nu_p(\alpha) = n\) where \(\alpha = p^nu\) for some \(u \in \Units\). This
function is called the \emph{\(p\)-adic valuation} of \(\alpha\). We write
\(\nu_p(0) = +\infty\), and this valuation satisfies the following properties
for all \(\alpha, \beta \in \Integers_p\):

\smallskip

\begin{enumerate}[nosep, label=(\roman*)]
    \item \(\nu_p(\alpha) =\infty\) if and only if \(\alpha =
    0\);\label{item:padic-valuation-zero}
    \item \(\nu_p(\alpha) = 0\) if and only if \(\alpha \in
    \Units\);\label{item:padic-valuation-units}
    \item \(\nu_p(\alpha\beta) = \nu_p(\alpha) + \nu_p(\beta)\);
    and\label{item:padic-valuation-product}
    \item \(\nu_p(\alpha + \beta) \geq \min\{\nu_p(\alpha), \nu_p(\beta)\}\).
    \label{item:padic-valuation-sum}
\end{enumerate}

\smallskip

Properties \ref{item:padic-valuation-zero} and \ref{item:padic-valuation-units}
are immediate from the definition of \(\nu_p\). The product formula in
\ref{item:padic-valuation-product} follows trivially if either \(\alpha\) or
\(\beta\) is zero. If that is not the case, suppose that \(\nu_p(\alpha) = m\)
and \(\nu_p(\beta) = n\) for some integers \(m\) and \(n\). We then have
\(\alpha = p^mu\) and \(\beta = p^nv\) for some \(u, v \in \Units\). It follows
that \(\alpha\beta = p^{m + n}uv\) so that the product \(\alpha\beta\) lies in
the image of the map \([p^{m+n}]\) defined in
Theorem~\ref{thm:exact-sequence-zp} and \(\nu_p(\alpha\beta) \geq m + n\). On
the other hand, both \(\pi_m(\alpha)\) and \(\pi_n(\beta)\) are nonzero and thus
\(\pi_{m+n}(\alpha\beta)\) is nonzero and thus \(\nu_p(\alpha\beta) \leq m +
n\). This proves the product formula.

Now to prove \ref{item:padic-valuation-sum}, we can assume without loss of
generality that \(m \leq n\), so that the image of \([p^n]\) is contained in
\([p^m]\) and thus both \(\alpha\) and \(\beta\) lie in the image of \([p^m]\),
as does \(\alpha + \beta\), and thus \(nu_p(\alpha + \beta) \geq m =
\min\{\nu_p(\alpha), \nu_p(\beta)\}\).

From this we conclude that \(\Integers_p\) has no zero divisors and is therefore
an integral domain. Indeed if \(\alpha\) and \(\beta\) are nonzero \(p\)-adic
integers, we have \(\nu_p(\alpha\beta) = \nu_p(\alpha) + \nu_p(\beta) < \infty\)
and thus \(\alpha\beta \neq 0\).

\subsection{The \(p\)-adic absolute value and
    metric.}\label{sec:padic-absolute-value}~Because \(\Integers_p\) is an
    integral domain, we can define the field of fractions\footnote{ Let \(R\) be
    an integral domain. The \emph{field of fractions} of \(R\) is the set of
    equivalence classes of pairs \((a, b)\) where \(a, b \in R\) and \(b \neq
    0\) under the equivalence relation \((a, b) \sim (c, d)\) if and only if
    \(ad = bc\). We denote the field of fractions of \(R\) by \(\FracField(R)\).
    See also \cite[p.~142]{hungerford2012algebra}. This definition generalizes
    the construction of the field of rational numbers from the ring of integers.
    } of \(\Integers_p\), which we shall denote by \(\Rationals_p\). The
    elements of \(\Rationals_p\) are called \emph{\(p\)-adic numbers}. The
    valuation \(\nu_p\) also allows us to define the \(p\)-adic absolute value
    \(|\ |_p : \Rationals_p \to \Reals_{\geq 0}\) by
\[
  |\alpha|_p = p^{-\nu_p(\alpha)},
\]
for all \(\alpha \in \Rationals_p\), with equality if and only if \(\alpha =
0\). Recall that an absolute value function on a field \(\field\) is any
function \(|\ | : \field \to \Reals_{\geq 0}\) satisfying the following
properties for all \(x, y \in \field\):

\smallskip

\begin{enumerate}[nosep, label=(\roman*)]
    \item \(|x| = 0\) if and only if \(x = 0\);\label{item:abs-val-zero}
    \item \(|xy| = |x||y|\);\label{item:abs-val-product}
    \item \(|x + y| \leq |x| + |y|\).\label{item:abs-val-sum}
\end{enumerate}

The reader may verify that the \(p\)-adic absolute value we have just defined
satisfies the above properties. Moreover, we also claim that the \(p\)-adic
absolute value is \emph{non-archimedean}, i.e., it satisfies the stronger
property that \(|x + y|_p \leq \max\{|x|_p, |y|_p\}\) for all \(x, y \in
\Rationals_p\). To see this, let \(x, y \in \Rationals_p\) and suppose, without
loss of generality, that \(|x|_p \geq |y|_p\). Then \(p^{-\nu_p(x)} \geq
p^{-\nu_p(y)}\) which implies that \(p^{\nu_p(x)} \leq p^{\nu_p(y)}\) and hence
\(\min\{\nu_p(x), \nu_p(y)\} = \nu_p(x)\). It follows that \(p^{-\nu_p(x + y)}
\leq p^{-\nu_p(x)}\) or equivalently, \(|x + y|_p \leq |x|_p = \max\{|x|_p,
|y|_p\}\), proving our claim.

Our definition of the \(p\)-adic absolute value provides us with a notion of
    ``size'' in \(\Rationals_p\). With this, we can now introduce the notion of
    ``distance'' between two \(p\)-adic numbers by defining a metric\footnote{
    Again for completeness we recall the definition of a metric. Let \(X\) be a
    set. A \emph{metric} on \(X\) is a function \(d : X \times X \to
    \Reals_{\geq 0}\) satisfying the following properties for all \(x, y, z \in
    X\):
    \begin{enumerate}[nosep, label=(\roman*)]
        \item \(d(x, y) = 0\) if and only if \(x = y\);\label{item:metric-zero}
        \item \(d(x, y) = d(y, x)\);\label{item:metric-symmetry}
        \item \(d(x, z) \leq d(x, y) + d(y, z)\).\label{item:metric-triangle}
    \end{enumerate}
    A set \(X\) equipped with a metric \(d\) is called a \emph{metric space}.
See also \cite[p.~30]{rudin1976principles}. } on \(\Rationals_p\). The
\(p\)-adic absolute value induces the distance function
\[
  d(\alpha, \beta) = |\alpha - \beta|_p.
\]

The reader may verify that this is indeed a metric on \(\Integers_p\) using the
definition we have provided in the footnote \emph{infra}.

\subsection{Completions.}~Having defined the \(p\)-adic absolute value and
metric, we shall now use them to introduce another way of looking at the field
of \(p\)-adic numbers. The main goal of this section is to prove that the field
of \(p\)-adic numbers is the completion of the field of rational numbers with
respect to the \(p\)-adic absolute value. The treatment throughout this paper
has been mainly algebraic, but we shall now introduce some results from analysis
to achieve our aim. Nevertheless, the exposition in this section shall be
cursory, as we shall not need the full strength of the results we shall prove
here. We refer the reader to Chapters 2--3 of \cite{gouvea1997p} for a more
detailed treatment.

We first recall the following definitions from analysis; the reader may consult
\cite{rudin1976principles} for a more detailed treatment. For these definitions,
let \(\field\) be a field and let \(|\ |\) be an absolute value on \(\field\). 

\begin{enumerate}[nosep, label=(\roman*)]
    \item A sequence \((x_n)\) in \(\field\) is a map \(x_n : \Naturals \to
    \field\).
    \item A sequence \((x_n)\) in \(\field\) is \emph{Cauchy} if for every
    \(\epsilon > 0\) there exists an integer \(N\) such that \(|x_n - x_m| <
    \epsilon\) for all \(n, m \geq N\).
    \item A sequence \((x_n)\) in \(\field\) is \emph{convergent} if there
    exists an element \(x \in \field\) such that for every \(\epsilon > 0\)
    there exists an integer \(N\) such that \(|x_n - x| < \epsilon\) for all \(n
    \geq N\).
    \item A field \(\field\) is \emph{complete} with respect to \(|\ |\) if
    every Cauchy sequence in \(\field\) is convergent.
    \item An \emph{open ball} in \(\field\) is a set of the form \(B(x,\epsilon)
    = \{y \in \field : |x - y| < \epsilon\}\) for some \(x \in \field\) and
    \(\epsilon > 0\).
    \item A subset \(E\) of \(\field\) is \emph{dense} in \(\field\) if every
    element of \(\field\) is a limit of a sequence of elements of \(E\).
    Alternatively, \(E\) is dense in \(\field\) if every open ball in \(\field\)
    contains an element of \(E\).
\end{enumerate}

We shall now start to develop the results we need for the main theorem of this
section. Let \(\field\) be a field and let \(|\ |_1\) and \(|\ |_2\) be two
absolute values on \(\field\). We say that \(|\ |_1\) and \(|\ |_2\) are
\emph{equivalent} if they define the same topology on \(\field\), i.e., if every
open ball in \(\field\) with respect to \(|\ |_1\) is also an open ball with
respect to \(|\ |_2\) and \emph{vice versa.} This brings us to the following
lemma.

\begin{lemma}
    Let \(\field\) be a field and let \(|\ |_1\) and \(|\ |_2\) be two absolute
    values on \(\field\). The following statements are equivalent:

    \smallskip

    \begin{enumerate}[nosep, label=(\alph*)]
        \item The absolute values \(|\ |_1\) and \(|\ |_2\) are equivalent.
        \item Any sequence in \(\field\) converging with respect to \(|\ |_1\)
        to some limit \(a \in \field\) also converges with respect to \(|\ |_2\)
        to \(a\).
        \item For any \(x \in \field\), \(|x|_1 < 1\) if and only if \(|x|_2 <
        1\).
        \item There exists a positive real number \(c\) such that \(|x|_1 =
        |x|_2^c\) for all \(x \in \field\).
    \end{enumerate}
\end{lemma}

See \cite[pp.~54-56]{gouvea1997p} for a proof. The lemma above helps us prove
the following theorem, due to Ostrowski.

\begin{theoremx}[Ostrowski]\label{thm:ostrowski} Every non-trivial absolute
    value on \(\Rationals\) is equivalent to either the usual absolute value
    \(|\ |_{\infty}\) or the \(p\)-adic absolute value \(|\ |_p\) for some prime
    number \(p\).
\end{theoremx}

By the `trivial' absolute value we mean the absolute value which is defined as
\(1\) if \(x \neq 0\) and \(0\) otherwise. A proof of this theorem can again be
found in \cite[pp.~56--59]{gouvea1997p}.

Ostrowski's theorem help us make sense of what is convention among number
theorists and what will be our practice in this essay of often referring to the
set of primes ``including the symbol \(\infty\)''.\footnote{In
\S\,\ref{sec:hilbert-symbol-intro}, we provide a shorthand notation for the set
of primes including the symbol \(\infty\), which we shall call the set of
\emph{places} of \(\Rationals\).} By this theorem we can think of all
non-trivial absolutes values on \(\Rationals\) as being prime, both in the sense
of the \(p\)-adic absolute value for some prime \(p\) and in the sense of the
usual absolute value for the ``prime at infinity.'' On the other hand,
Ostrowski's theorem also reduces any result about absolute values on
\(\Rationals\) to the case of the \(p\)-adic absolute value for some prime \(p\)
and the usual absolute value, as we have in the following lemma.

\begin{lemmax}\label{lem:ostrowski-lemma}
    {\normalfont\cite[pp.~63--64]{gouvea1997p}} The field \(\Rationals\) is not
    complete with respect to any of its non-trivial absolute values.
\end{lemmax}

Because \(\Rationals\) is not complete with respect to any of its non-trivial
absolute values, we shall now construct its completion. We will follow the path
of least resistance \cite[p.~64]{gouvea1997p} and add to \(\Rationals\) the
limits of all the Cauchy sequences. Since those limits do not currently exist,
we need to build them using the limits of the Cauchy sequences we already have.
We now make this construction precise. We shall denote by \(\mathscr{C}\) (or
\(\mathscr{C}_p\) if we want to emphasize \(p\)) the set of all Cauchy sequences
of elements of \(\Rationals\). If \((x_n)\) and \((y_n)\) are sequences in
\(\mathscr{C}\), then defining \((x_n) + (y_n) = (x_n + y_n)\) and \((x_n)(y_n)
= (x_ny_n)\) makes \(\mathscr{C}\) into a commutative ring with identity (the
constant sequence \((1, 1, 1, \dots)\)).

We are interested in a specific subset of \(\mathscr{C}\), to wit, the subset of
sequences that tend to zero. Denote this set by \(\mathfrak{m}\) for brevity. By
our definition of multiplication in \(\mathscr{C}\), it is clear that this
subset is an ideal of \(\mathscr{C}\). That is, if \((x_n)\) is a sequence in
\(\mathscr{C}\) and \((y_n)\) is a sequence that tends to zero, then
\((x_n)(y_n) = (x_ny_n)\) is a sequence that tends to zero. We also claim that
this ideal is maximal, i.e., if \(\mathfrak{a}\) is an ideal of \(\mathscr{C}\)
such that \(\mathfrak{m} \subseteq \mathfrak{a}\), then \(\mathfrak{a} =
\mathfrak{m}\). This latter claim is a bit trickier to prove. Since the constant
sequence \((1, 1, 1, \dots)\) which is the identity in \(\mathscr{C}\) converges
to \(1\), it follows that \((1, 1, 1, \dots) \notin \mathfrak{m}\) and thus
\(\mathfrak{m} \subsetneq \mathscr{C}\).

Now to show that \(\mathfrak{m}\) is indeed maximal, suppose first that
\(\mathfrak{a}\) is an ideal of \(\mathscr{C}\) such that \(\mathfrak{m}
\subsetneq \mathfrak{a} \subseteq \mathscr{C}\). Suppose \((x_n) \in
\mathfrak{a} \setminus \mathfrak{m}\); then since the limit of \((x_n)\) is not
zero, there must exist some number \(\gamma > 0\) and an integer \(N\) such that
\(|x_n| \geq \gamma\) for all \(n \geq N\). This would imply that \(x_n \neq 0\)
for all \(n \geq N\) so that we can define a new sequence \((y_n)\) by setting
\(y_n = 1/x_n\) for all \(n \geq N\) and \(y_n = 0\) otherwise. Then we have
\[
    |y_{n+1} - y_n| = \left|\frac{1}{x_{n+1}} - \frac{1}{x_n}\right| = \frac{|x_n - x_{n+1}|}{|x_nx_{n+1}|} \leq \frac{1}{\gamma^2}|x_n - x_{n+1}| \to 0
\]
and thus \((y_n)\) is a Cauchy sequence. Now let \((z_n)\) be the sequence
defined by \(z_n = x_ny_n\), so that \(z_n = 0\) for all \(n < N\) and \(z_n =
1\) for all \(n \geq N\). If we define a new sequence \((w_n)\) by subtracting
\((z_n)\) from the constant sequence \((1, 1, 1, \dots)\), we obtain a sequence
that tends to zero. But since \((w_n)\) and \((z_n)\) are both in
\(\mathfrak{a}\), it follows that \((1, 1, 1, \dots)\) which is the identity in
\(\mathscr{C}\) is also in \(\mathfrak{a}\) and thus \(\mathfrak{a} =
\mathscr{C}\). This proves that \(\mathfrak{m}\) is maximal.

Having laid the groundwork, we can now define the \emph{completion} of
\(\Rationals\) with respect to the \(p\)-adic absolute value as the quotient
ring
\begin{equation}\label{eq:completion-of-rationals}
    \Rationals_p = \mathscr{C}_p / \mathfrak{m}_p;
\end{equation}
moreover, since \(\mathfrak{m}_p\) is maximal, \(\Rationals_p\) is a field. We
shall now prove that \(\Rationals_p\) is indeed the completion of \(\Rationals\)
with respect to the \(p\)-adic absolute value.

The elements of \(\Rationals_p\) are equivalence classes of Cauchy sequences of
rational numbers. First observe that for any rational number \(x\), the constant
sequence \((x, x, x, \dots)\) is a Cauchy sequence and thus defines an element
of \(\mathscr{C}\). The map \(x \mapsto (x, x, x, \dots)\) is an embedding of
\(\Rationals\) into \(\mathscr{C}\) since two sequences are equal if and only if
every term of one sequence is equal to the corresponding term of the other
sequence, which for two constant sequences \((x, x, x, \dots)\) and \((y, y, y,
\dots)\) is equivalent to saying that \(x = y\). By our definition of
\(\Rationals_p\) in \eqref{eq:completion-of-rationals} above, one can observe
that two sequences \((x_n)\) and \((y_n)\) in \(\mathscr{C}_p\) differ by an
element of \(\mathfrak{m}_p\) if and only if their difference \((x_n - y_n)\) is
a sequence that tends to zero. This is equivalent to saying that we maintain the
inclusion \(\Rationals \hookrightarrow \Rationals_p\), which sends each element
\(x \in \Rationals\) to the equivalence class of the constant sequence \((x, x,
x, \dots)\). Now to verify that we have indeed obtained a completion of
\(\Rationals\), we check two things: first that \(\Rationals\) is dense in
\(\Rationals_p\) and second that \(\Rationals_p\) is complete with respect to
the \(p\)-adic absolute value.

\begin{lemmax}
    {\normalfont\cite[pp.~68--69]{gouvea1997p}} The image of \(\Rationals\) in
    \(\Rationals_p\) is a dense subset of \(\Rationals_p\).
\end{lemmax}

\begin{proof}
    We begin with a definition. We say that a Cauchy sequence \((x_n)\)
    represents an element \(\alpha \in \Rationals_p\) if \(|x_n|_p \to
    |\alpha|_p\). Our goal is to show that every open ball around an element
    \(\alpha\) of \(\Rationals_p\) contains an element of the image of
    \(\Rationals\), i.e., a constant sequence.
    
    Fix a radius \(\epsilon > 0\). Let \((x_n)\) be a Cauchy sequence
    representing \(\alpha\) and let \(\epsilon' > 0\) be another number such
    that \(\epsilon > \epsilon'\). Since \((x_n)\) is Cauchy, the exists an
    \(N\) such that \(|x_n - x_m|_p < \epsilon'\) for all \(n, m \geq N\). Let
    \(y = x_N\) and consider the constant sequence \((y, y, y, \dots)\). We
    claim that this sequence represents an element of \(\Rationals_p\) which is
    in the open ball \(B(\alpha, \epsilon)\). To see this, observe that \(\alpha
    - y\) is represented by the sequence \((x_n - y)\) and thus \(|\alpha - y|_p
    = \lim_{n \to \infty}|x_n - y|_p\). But then for any \(n \geq N\), we have
    \[|x_n - y|_p = |x_n - x_N|_p < \epsilon' < \epsilon\] and thus the constant
    sequence \((y, y, y, \dots)\) is in the open ball \(B(\alpha, \epsilon)\).
\end{proof}

\begin{lemmax}
    {\normalfont\cite[pp.~69--70]{gouvea1997p}} The field \(\Rationals_p\) is
    complete with respect to the \(p\)-adic absolute value.
\end{lemmax}

See \cite[pp.~69--70]{gouvea1997p} for a proof.

Putting this altogether, we are now able to prove the main result of this
section.

\begin{theoremx}
    {\normalfont\cite[pp.~70--71]{gouvea1997p}} For each prime number \(p\),
    there exists a field \(\Rationals_p\) satisfying the following properties:

    \smallskip

    \begin{enumerate}[nosep, label=(\alph*)]
        \item There is an embedding \(\Rationals \hookrightarrow \Rationals_p\)
        such that the image of \(\Rationals\) is dense in \(\Rationals_p\).
        \item The field \(\Rationals_p\) is complete with respect to the
        \(p\)-adic absolute value.
    \end{enumerate}

    \smallskip

    Moreover, \(\Rationals_p\) is unique up to isomorphism.
\end{theoremx}

\begin{proof}
    It remains to prove that the field \(\Rationals_p\) we have constructed is
    unique up to isomorphism. Suppose another field \(K\) exists satisfying the
    conditions of the theorem. Then there exists an embedding \(\Rationals
    \hookrightarrow K\) preserving the \(p\)-adic absolute value and such that
    the image of \(\Rationals\) is dense in \(K\). Moreover, for any Cauchy
    sequence \((x_n)\) in \(\Rationals\), the image of \((x_n)\) in \(K\) is
    also Cauchy and thus converges to some element of \(K\). Now take some
    element \(\alpha \in \Rationals_p\). Then there exists a Cauchy sequence
    \((x_n)\) in \(\Rationals\) representing \(\alpha\). Since \(x_n\) is a
    rational number, we can find its image in \(K\), which by our definition of
    \(K\) is also a Cauchy sequence and thus converges to some element of \(K\)
    because \(K\) is complete. Call this limit in \(K\), \(f(\alpha)\), so that
    the function \(f: \Rationals_p \to K\) is the identity on \(\Rationals\) and
    is an isomorphism. Finally, its uniqueness follows from the factthat it
    induces the identity on \(\Rationals\).
\end{proof}

This theorem allows us to identify the field of \(p\)-adic numbers with the
completion of the field of rational numbers with respect to the \(p\)-adic
absolute value without regard to the construction we have provided above, but
instead by simply invoking the properties we have identified.

% Finally, the ring \(\Integers_p\) is complete with respect to this metric and
% \(\Integers\) is dense in \(\Integers_p\). \cite[p.~12]{serre2012course}


\subsection{Polynomials over \(\Integers_p\) and \(\Rationals_p\).}~We shall now
consider polynomials with coefficients in \(\Integers_p\). We are interested, as
usual, in finding the roots of some polynomial \(f(x)\) in \(\Integers_p[x]\),
i.e., we want to find the \(p\)-adic integers (or \(p\)-adic numbers) \(\alpha\)
such that \(f(\alpha) = 0\). We can observe, however, that finding a solution in
\(\Rationals_p\) for some prime \(p\) does not necessarily guarantee the
existence of solutions in \(\Rationals\). For example, the polynomial \(x^2 -
2\) has roots in \(\Rationals_7\) but clearly \(\sqrt{2} \notin \Rationals\). We
shall defer answering the question of when we can find solutions in
\(\Rationals\) to later in this chapter; for now we shall investigate the
existence of solutions in \(\Integers_p\) and
\(\Rationals_p\).\label{sec:polynomials-over-zp}

We shall again denote by \(A_n\) the ring \(\Integers / p^n \Integers\) of
integers modulo \(p^n\). If \(f\) is a polynomial in \(\Integers_p[x]\) then for
any \(n \geq 1\), we write \(f_n\) for the reduction of \(f\) modulo \(p^n\),
i.e., \(f_n(x) = f(x) \pmod{p^n}\).

We begin with a lemma.

\begin{lemma}
    {\normalfont\cite[p.~13]{serre2012course}} If \(\dots \to D_n \to D_{n - 1}
    \to \dots \to D_1\) is a projective system (as in {\normalfont
    \S\,\ref{sec:padic-def}}) and \(D = \varprojlim D_n\), then \(D\) is
    nonempty if each \(D_n\) is nonempty.
\end{lemma}

\begin{proof}
    If \(D_n \to D_{n-1}\) is surjective, then \(D_{n-1}\) is nonempty if
    \(D_n\) is nonempty, and the lemma follows by induction. We thus reduce the
    lemma to this special case. Let \(D_{n, p}\) be the image of \(D_{n+p}\) in
    \(D_n\). Fix \(n\) so that the \(D_{n,p}\) form a decreasing family of
    finite, nonempty subsets (i.e., \(D_{n,p+1} \subseteq D_{n,p}\) and
    \(D_{n,p}\) is nonempty for all \(p\)). This family is \emph{stationary},
    i.e., \(D_{n,p} = D_{n,p+1}\) for all \(p\) sufficiently large. Let \(E_n\)
    be the limit value of the \(D_{n,p}\) as \(p\) tends to infinity. Then \(D_n
    \to D_{n-1}\) carries \(E_n\) onto \(E_{n-1}\) (i.e., \(E_n =
    \phi_n(E_{n-1})\)). Since the \(E_n\) are nonempty, we have \(\varprojlim
    E_n \neq \emptyset\) and the lemma follows.
\end{proof}

\begin{theoremx}\label{thm:zeros-of-polynomials-in-zp} Let \(\mathfrak{F}\) be a
    family of polynomials in \(m\) variables with coefficients in
    \(\Integers_p\). The following statements are equivalent:

    \smallskip

    \begin{enumerate}[nosep, label=(\alph*)]
        \item All the polynomials in \(\mathfrak{F}\) have a common root in
        \(\Integers_p^m\).
        \item For all \(n > 1\), all the polynomials \(f_n\) for each \(f \in
        \mathfrak{F}\) have a common root in \(A_n^m\).
    \end{enumerate}
\end{theoremx}

\begin{proof}
    Let \(R\) (resp. \(R_n\)) be the set of common roots of the polynomials in
    \(\mathfrak{F}\) in \(\Integers_p^m\) (resp. \(A_n^m\)). The \(R_n\) are
    finite and we have \(R = \varprojlim R_n\). The theorem follows from the
    lemma.
\end{proof}

We say that an element \(x = (x_1, \dots, x_m)\) of \(\Integers_p^m\) is
primitive if at least one of the \(x_i\) is invertible, i.e., not all of the
\(x_i\) are divisible by \(p\). We define similarly the notion of a primitive
element of \(A_n^m\). We now relate Theorem \ref{thm:zeros-of-polynomials-in-zp}
to solutions in \(\Rationals_p\).

\begin{theoremx}\label{thm:zeros-of-polynomials-in-qp-family} Let
    \(\mathfrak{F}\) be a family of homogeneous polynomials in \(m\) variables
    with coefficients in \(\Integers_p\). Then the following statements are
    equivalent:

    \smallskip

    \begin{enumerate}[nosep, label=(\alph*)]
        \item The polynomials in \(\mathfrak{F}\) have a nontrivial common root
        in \(\Rationals_p^m\).
        \item The polynomials in \(\mathfrak{F}\) have a common primitive root
        in \(\Integers_p^m\).
        \item For all \(n > 1\), the polynomials \(f_n\) for each \(f \in
        \mathfrak{F}\) have a common primitive root in \(A_n^m\).
    \end{enumerate}
\end{theoremx}

\begin{proof}
    The equivalence (b) \(\iff\) (c) follows from the lemma above. We prove (a)
    \(\iff\) (b). If \(x\) is a common primitive root of the polynomials in
    \(\mathfrak{F}\) in \(\Integers_p^m\), then \(x \neq 0\) and is also a
    common root of the polynomials in \(\mathfrak{F}\) in \(\Rationals_p^m\).
    Conversely, let \(x = (x_1, \dots, x_m)\) be a nontrivial common root of the
    polynomials in \(\mathfrak{F}\) in \(\Rationals_p^m\) and write
    \[
      h = \inf\{\nu_p(x_i) : 1 \leq i \leq m\}.
    \]
    Let \(y = p^{-h}x\) so that \(y\) is a primitive element of
    \(\Integers_p^m\). Moreover, \(y\) is a common root of the polynomials in
    \(\mathfrak{F}\) in \(\Integers_p^m\).
\end{proof}

The theorems in this section can be used to show that a polynomial in has a root
in \(\Integers_p\) by proving that it has a solution in \(A_n\).
\cite{weismann2006annotations}

\subsection{Hensel's lemma.}~We now introduce an important result, due to
Hensel, which allows us to pass from a solution modulo \(p^n\) to a ``true''
solution in \(\Integers_p\). In essence, this result provides a meaningful way
to approximate roots of polynomials in \(\Integers_p[x]\). Finding the root of a
polynomial modulo \(p^n\) is the same as finding a value which is within the
radius \(p^{-n}\) of the solution in the \(p\)-adic metric
\cite{weismann2006annotations} and thus the goal of this approximation algorithm
is to make \(n\) successively larger (and thus the radius \(p^{-n}\)
successively smaller) until we have a solution in \(\Integers_p\). This result,
which we shall call Hensel's lemma, is similar to Newton's method in calculus
for finding roots of polynomials given some initial
approximation.\footnote{Newton's method is indeed a special case of Hensel's
lemma. See \cite{von1984hensel}.}

\begin{lemma}
    Let \(f \in \Integers_p[x]\) and \(f'\) be its derivative. If \(n\) and
    \(k\) are integers satisfying \(0 \leq 2k < n\), and for any \(\alpha \in
    \Integers_p\) we have \(f(\alpha) \equiv 0 \pmod{p^n}\) and
    \(\nu_p(f'(\alpha)) = k\), then there exists \(\beta \in \Integers_p\) such
    that \(f(\beta) \equiv 0 \pmod{p^{n + 1}}\), \(\nu_p(f'(\beta)) = k\), and
    \(\beta \equiv \alpha \pmod{p^{n-k}}\).
\end{lemma}

\begin{proof}
    We follow the proof in \cite{serre2012course}. Take \(\beta\) of the form
    \(\alpha + p^{n-k}\gamma\) with \(\gamma \in \Integers_p\). By Taylor's
    formula we have
    \[
        f(\beta) = f(\alpha) + p^{n-k}f'(\alpha)\gamma + p^{2n-2k}\zeta
    \]
    with \(\zeta \in \Integers_p\). Then by our hypothesis \(f(\alpha) =
    p^n\eta\) and \(f'(\alpha) = p^k\xi\) with \(\eta \in \Integers_p\) and
    \(\xi \in \Units\). This alows us to choose \(\gamma\) such that
    \[
        \eta + \gamma\xi \equiv 0 \pmod{p}.
    \]
    From this we have
    \[
        f(\beta) = p^n(\eta + \gamma\xi) + p^{2n-2k}\zeta \equiv 0 \pmod{p^{n+1}}
    \]
    since \(2n - 2k > n\). Finally, applying Taylor's formula to \(f'\) shows
    that \(f'(\beta) \equiv p^k \xi \pmod{p^{n-k}}\); since \(n - k > k\), we
    have \(\nu_p(f'(\beta)) = k\).
\end{proof}

\medskip

We can extend this result to polynomials in several variables.

\begin{theorem}
    Let \(f\) be a polynomial in \(m\) variables with coefficients in
    \(\Integers_p\) and let \(j\) be an integer such that \(0 \leq j \leq m\).
    If \(n\) and \(k\) are integers with \(0 \leq 2k < n\) and if \(f'(\alpha)
    \equiv 0 \pmod{p^n}\) and
    \[
        \nu_p\left(\frac{\partial f}{\partial a_j}(\alpha)\right) = k
    \]
    for some \(\alpha = (a_i) \in \Integers_p^m\), then there exists a zero
    \(\beta = (b_i) \in \Integers_p^m\) of \(f\) such that
    \[
        \alpha \equiv \beta \pmod{p^{n-k}}.
    \]
\end{theorem}

See \cite[pp.~14--15]{serre2012course} for a proof.

\medskip

By using Hensel's lemma, one can ``lift'' (i.e., obtain) from a given root \(r\)
of a polynomial \(f\) modulo \(p^k\) another root \(s\) of \(f\) modulo
\(p^{k+1}\) such that \(s \equiv r \pmod{p^k}\). The above theorem is often
applied as the following corollary.

\begin{corollary}
    Every simple root of reduction modulo \(p\) of a polynomial \(f\) can be
    lifted to a root of \(f\) in \(\Integers_p[x]\).
\end{corollary}

Here a root \(x = (x_1, \dots, x_n)\) is considered ``simple'' if \(f(x) = 0\)
but \(\partial f / \partial x_i \neq 0\) for some \(i\). This is a special case
of the theorem with \(n = 1\) and \(k = 0\).

\smallskip

Let us now use Hensel's lemma to solve the congruence
\[
    x^2 \equiv 2 \pmod{7^3}.  
\]
The case \(x^2 \equiv 2 \pmod{7}\). This has solutions \(x \equiv 3\) and \(x
\equiv 4 \pmod{7}\). Consider first the solution \(x \equiv 3 \pmod{7}\); then
\(x = 3 + 7y\) for any integer \(y\). Substituting this into the congruence
gives
\[
    9 + 42y + 49y^2 \equiv 0 \pmod{7^2},
\]
or equivalently,
\[
    7(1 + 6y) \equiv 1 + 6y \equiv 0 \pmod{7^2}.
\]
Thus \(y \equiv 1 \pmod{7}\) and we obtain \(x \equiv 10 \pmod{7^2}\).
Similarly, for \(x \equiv 4 \pmod{7}\), we get \(x \equiv 39 \equiv -10
\pmod{7^2}\).

Now let us solve the initial congruence modulo \(7^3\). Consider again the case
\(x \equiv 10 \pmod{7^2}\). Then \(x = 10 + 7^2z\) for some integer \(z\).
Substituting this into the congruence gives
\[
    (10+7^2z)^2 \equiv 2 \pmod{7^3},
\]
and hence by routine computation we obtain \(z \equiv 2 \pmod{7}\) and thus \(x
\equiv 10 + 7^2z \equiv 108 \pmod{7^3}\). Similarly, for \(x \equiv 39
\pmod{7^2}\), we get \(x \equiv 235 \pmod{7^3}\).

Thus we were able to ``lift'' solutions modulo \(7^2\) given solutions modulo
\(7\). This process can be repeated indefinitely, and one can observe that the
solutions modulo \(x^2 \equiv 2 \pmod{7^k}\) follow a familiar pattern:

\begin{align*}
    k = 1 & \qquad \pm 3 \\
    k = 2 & \qquad \pm (3 + 7) \\
    k = 3 & \qquad \pm (3 + 7 + 2 \times 7^2) \\
    k = 4 & \qquad \pm (3 + 7 + 2 \times 7^2 + 6 \times 7^3) \\
\end{align*}

\section{The Hilbert symbol}\label{sec:hilbert-symbol}

We shall now review the norm-residue symbol introduced by Hilbert in
\cite[pp.~286--287]{hilbert1932theorie}. Our treatment in
\S\,\ref{sec:hilbert-symbol-intro} shall be mainly cursory as our goal is to use
the properties we shall develop here to simplify some of the arguments in our
proof of the Hasse-Minkowski theorem, which is the main result of this chapter.
After \S\,\ref{sec:hilbert-symbol-intro}, we shall then continue with the
development of the theory in keeping with the rigor of the rest of this paper.
We refer the reader to \cite{serre2012course} and to
\cite{sutherland2013introduction10} for the details of the omitted proofs.

\subsection{Definition and properties.}~Throughout this section we take
\(\field\) to be the field \(\Rationals_p\) or \(\Reals.\) For any \(a\) and
\(b\) in \(\fieldU\) we define the Hilbert norm-residue symbol (or more simply,
the Hilbert symbol) \((a, b)\) by
\[
    (a, b) = \begin{cases}
        \ \ 1 & \text{if } z^2 - ax^2 - by^2 = 0 \text{ has a nontrivial solution in } \field^3, \\
        \ \ -1 & \text{otherwise}.
    \end{cases}
\]
\label{sec:hilbert-symbol-intro}
Let \(\Places\) be the union of the set of all prime numbers and the symbol
\(\infty\); we shall call \(\mathfrak{V}\) the set of ``places'' in
\(\Rationals\). For each \(v \in \Places\), we write \((a, b)_v\) for  each \(a,
b \in \fieldU\) to denote the image of \((a, b)\) under the map \(\SquareClass
\times \SquareClass \to \{1, -1\}\) induced by the Hilbert symbol in
\(\Rationals_v\), where we define \(\Rationals_{\infty}\) to be \(\Reals\).

Because \((a, b)\) remains unchanged when \(a\) and \(b\) are multiplied by any
non-zero square we see that the Hilbert symbol defines a map \(\SquareClass
\times \SquareClass \to \{1, -1\}\). We formalize these observations in the
following lemma.
\begin{lemma}
    Let \(a, b \in \fieldU\). The following statements are equivalent:

    \medskip

    \begin{enumerate}[nosep, label=(\alph*)]
        \item \((a, b) = 1\);
        \item The equation \(z^2 - ax^2 - by^2 = 0\) has a nontrivial solution
        in \(\field^3\);
        \item The form \(z^2 - ax^2 - by^2\) represents zero in \(\field\).
        \item The element \(a\) is a norm in \(\field(\sqrt{b})\).
    \end{enumerate}
\end{lemma}

See \cite[pp.~17--18]{serre2012course} or
\cite[p.~1]{sutherland2013introduction10} for a proof. We now show some
elementary properties of the Hilbert symbol.

\begin{theorem}
    The Hilbert symbol satisfies the following properties for all \(a, b, c, a'
    \in \fieldU\), (with \(a \neq 1\) whenever \(1-a\) appears in the formula):

    \smallskip

    \begin{enumerate}[nosep, label=(\alph*)]
        \item \((a, b) = (b, a)\);
        \item \((a, c^2) = 1\);
        \item \((a, -a) = -1\) and \((a, 1-a) = 1\);
        \item if \((a, b) = 1\), then \((aa', b) = (a', b)\); and
        \item \((a, b) = (a, -ab) = (a, (1-a)b)\).\label{item:formula-e}
    \end{enumerate}
\end{theorem}

See \cite[pp.~19--21]{serre2012course} for a proof.

\subsection{Computation of the Hilbert symbol.}~We shall now study how to
explicitly compute the Hilbert symbol and show that it is bilinear and
nondegenerate. Before that, we first define the {Legendre symbol}. Let \(p\) be
an odd prime. An integer \(a\) is said to be a \emph{quadratic residue} modulo
\(p\) if there exists an integer \(x\) such that \(x^2 \equiv a \pmod{p}\). If
no such \(x\) exists, then \(a\) is said to be a \emph{quadratic nonresidue}
modulo \(p\). The \emph{Legendre symbol} \(\left(\frac{a}{p}\right)\) is defined
to be \(1\) if \(a\) is a quadratic residue modulo \(p\) and \(-1\) if \(a\) is
a quadratic nonresidue modulo \(p\). Legendre has provided the following
explicit formula for the symbol:\label{sec:computing-hilbert-symbol}
\[
       \left(\frac{a}{p}\right) = a^{(p-1)/2} \pmod{p}.  
\]
This brings us to the following theorem.
\begin{theorem}
    {\normalfont\cite[pp.~20--21]{serre2012course}} If \(\field = \Reals\), then
    \((a, b) = -1\) if both \(a\) and \(b\) are negative and \((a, b) = 1\)
    otherwise. If \(\field = \Rationals_p\), then, writing \(a\) and \(b\) as
    \(a = p^\alpha u\) and \(b = p^\beta v\) with \(u, v \in \Units\), we have
    \[
        (a, b) = (-1)^{\alpha\beta\epsilon(p)}\left(\frac{u}{p}\right)^{\beta}\left(\frac{v}{p}\right)^{\alpha},
    \]
    whenever \(p\) is an odd prime, and 
    \[
        (a, b) = (-1)^{\epsilon(u)\epsilon(v) + \alpha\omega(v) + \beta\omega(u)}
    \]
    whenever \(p = 2\). Here we denote by \(\epsilon(x)\) and \(\omega(x)\) the
    images in \(\Integers/2\Integers\) of the integers \((x-1)/2\) and
    \((x^2-1)/8\), respectively.
\end{theorem}

\emph{Proof}. We first prove the theorem for \(\field = \Reals\). We can suppose
that \(a, b \in \{\pm 1\}\) since \(\{\pm 1\}\) is a complete set of
representatives of \(\Reals^{\times}/(\Reals^{\times})^2\). By the theorem of
the previous section, if either \(a\) or \(b\) is equal to \(1\) then \((a,
b)_{\infty} = 1\) and by the lemma of the same theorem, \((-1, -1)_{\infty} =
-1\) since \(-1\) is not a norm in \(\Complex = \Reals(\sqrt{-1})\). See also
\cite[Theorem~10.4]{sutherland2013introduction10}.

Now let \(\field = \Rationals_p\) and let \(p\) be an odd prime. Since \((a,
b)_p\) depends only on the square classes of \(a\) and \(b\), we can assume that
\(\alpha, \beta \in \Integers/ 2\Integers\). We need two lemmas.

\begin{lemmax}\label{lem:serre-sol-zp}
    {\normalfont\cite[p.~21]{serre2012course}} Let \(v\) be a \(p\)-adic unit.
    If the equation \(z^2 - px^2 - vy^2 = 0\) has a nontrivial solution in
    \(\Rationals_p^3\), then it has a solution \((z, x, y)\) with \(z, y \in
    \Units\) and \(x \in \Integers_p\).
\end{lemmax}

\begin{proof}
    By Theorem\,\ref{thm:zeros-of-polynomials-in-qp-family} the given equation
    has a primitive solution \((z, y, x)\). We want to show that this solution
    can be chosen so that \(z, y \in \Units\) and \(x \in \Integers_p\). Suppose
    for the sake of contradiction that this solution does not satisfy the
    conditions of the theorem. Then \(z\) or \(y\) is divisible by \(p\); since
    \(z^2 - vy^2 \equiv 0 \pmod{p}\) and \(v \not\equiv 0 \pmod{p}\), we would
    have both \(z\) and \(y\) divisible by \(p\), hence \(px^2\) would be
    divisible by \(p^2\) and \(x\) would be divisible by \(p\). But this
    contradicts the fact that \((z, y, x)\) is a primitive solution.
\end{proof}

\begin{lemmax}\label{lem:hilbert-odd-units}
    {\normalfont\cite[Lemma~10.5]{sutherland2013introduction10}} If \(p\) is odd
    then \((u, v)_p = 1\) for all \(u, v \in \Units\).
\end{lemmax}

\begin{proof}
    The proof uses the Chevalley-Warning theorem, which states that in a finite
    field if the number of variables of a polynomial is greater than its degree,
    then the cardinality of its solution set (i.e., the number of zeroes it has)
    is divisible by the characteristic of the field. We have not taken up this
    result in this paper but the reader may refer to
    \cite[p.~5]{serre2012course} for a proof. As a corollary of this result, we
    have that all quadratic forms of rank \(3\) has a non-trivial solution
    modulo \(p\). Hensel's lemma can then be used to lift this solution to a
    solution in \(\Integers_p\).
\end{proof}

\bigskip

We now return to the proof of the theorem and consider three cases:

\smallskip

\begin{enumerate}[nosep, label=(\roman*), wide]
    \item If \(\alpha = \beta = 0\), then \((a, b)_p = (u, v)_p = 1\) by
    Lemma\,\ref{lem:hilbert-odd-units} above.
    \item If \(\alpha = 1\) and \(\beta = 0\), we need to show that \((pu, v)_p
    ={ \legendre{v}{p}}\). Since \((u, v)_p = 1\) we have \((pu, v)_p = (p,v)\)
    by the theorem of \S\,\ref{sec:hilbert-symbol-intro} so that it suffices to
    check \((p, v)_p ={\legendre{v}{p}}\). If \(v\) is a square then we have
    \[
    (p, v)_p = (p, 1)_p = (1, p)_p = 1 = \legendre{v}{p}.
    \]
    If \(p\) is not a square then \(z^2 - px^2 - vy^2 = 0\) has no nontrivial
    solutions in modulo \(p\) and hence has no primitive solutions. Thus we have
    \[
    (p, v)_p = -1 = \legendre{v}{p}.
    \]
    \item If \(\alpha = \beta = 1\), we must show that
    \[
    (pu, pv)_p = (-1)^{\frac{p-1}{2}}\legendre{u}{p}\legendre{v}{p}.
    \]
    Applying formula \ref{item:formula-e} of the theorem of
    \S\,\ref{sec:hilbert-symbol-intro} we obtain
    \[
    (pu, pv)_p = (pu, -p^2uv)_p = (pu, -uv)_p.
    \]
    Applying the formula in the preceding case we finally obtain
    \[
    (pu, -uv)_p = \legendre{-uv}{p} =\legendre{-1}{p} \legendre{u}{p}\legendre{v}{p} = (-1)^{\frac{p-1}{2}}\legendre{u}{p}\legendre{v}{p}.
    \]
\end{enumerate}

This concludes the proof for odd primes. For \(p = 2\), we still have \(\alpha,
\beta\) in \(\Integers/2\Integers\) and we thus consider each of the three cases
above.

\smallskip

\begin{enumerate}[nosep, label=(\roman*), wide]
    \item If \(\alpha = \beta = 0\), we must check that \((u, v)_p = 1\) if
    \(u\) or \(v\) is congruent to \(1 \pmod{4}\) and \((u, v)_p\) otherwise.
    The first case implies \(u \equiv 1 \pmod{8}\) or \(u \equiv 5 \pmod{8}\).
    If \(u \equiv 1 \pmod{8}\) then \(u\) is a square modulo \(8\) and we have
    \((u, v)_p = 1\). Otherwise, if \(u \equiv 5 \pmod{8}\), we have \(u + 4v
    \equiv 1 \pmod{8}\) and there exists a unit \(w\) such that \(w^2 = u +
    4v\); the form \(z^2 - ux^2 - 4vy^2\) thus has the nontrivial solution \((w,
    1, 2)\) and we have \((u, v)_p = 1\). % TODO ADD REFERENCE
    For the second case suppose \(u \equiv v \equiv -1 \pmod{4}\). If
    \((z,x,y)\) is a primitive root of \(z^2 - ux^2 - vy^2 = 0\), then \(z^2 +
    x^2 + y^2 \equiv 0 \pmod{4}\). But then again the squares modulo \(4\) are
    \(0\) and \(1\) so that this earlier congruence implies \(z \equiv x \equiv
    y \equiv 0 \pmod{2}\), contradicting the fact that \((z, x, y)\) is a
    primitive solution. Thus we have \((u, v)_p = -1\).

    \item If \(\alpha = 1\) and \(\beta = 0\), we must check that
    \[ (2u, v)_p = (-1)^{\epsilon(u)\epsilon(v) + \omega(v)}. \] First we want
    to show that \((2, v)_p = (-1)^{\omega(v)}\), i.e, that \((2, v)_p = 1\) is
    equivalent to \(v \equiv \pm 1\pmod{8}\). By Lemma\,\ref{lem:serre-sol-zp}
    above, if \((2, v)_p = 1\) then there exists \(2\)-adic integers \(x\),
    \(y\) and \(z\) such that \(z^2 - 2x^2 - vy^2\) and \(y, z \not\equiv 0
    \pmod{2}\). This implies that \(y^2 = z^2 \equiv 1 \pmod{8}\) and hence \(1
    - 2x^2 - v \equiv 0 \pmod{8}\). But the only squares modulo \(8\) are \(0\),
    \(1\) and \(4\) so that \(v \equiv \pm 1 \pmod{8}\). Conversely, if \(v
    \equiv 1 \pmod{8}\) then \(v\) is a square and \((2, v)_p = 1\). If \(v
    \equiv -1 \pmod{8}\) then the equation \(z^2 - 2x^2 - vy^2 = 0\) has the
    nontrivial solution \((1, 1, 1)\) modulo \(8\); by Hensel's lemma, this
    solution can be lifted to a solution in \(\Integers_2\) and thus we have
    \((2, v)_p = 1\).

    Next we show that \((2u, v)_p= (2, v)_p(u,v)_p\). By the theorem of
    \S\,\ref{sec:hilbert-symbol-intro}, this holds if either of \((2,v)_p\) or
    \((u, v)_p\) is equal to \(1\). The remaining case is \((2, v)_p = (u, v)_p
    = -1\), implying that \(v \equiv 3 \pmod{8}\) and \(u \equiv 3 \pmod{8}\).
    Multiplying \(u\) and \(v\) by squares, we can suppose that \(u = -1\) and
    \(v = 3\) or \(u = 3\) and \(v = -5\). But the equations
    \[
    z^2 + 2x^2 - 3y^2 = 0 \qquad \text{and} \qquad z^2 - 6x^2 + 5y^2 = 0
    \]
    then both have the nontrivial solution \((1, 1, 1)\) and therefore \((2u,
    v)_p = 1\).

    \item Finally, if \(\alpha = \beta = 1\), we must show that
    \[
    (2u, 2v)_p = (-1)^{\epsilon(u)\epsilon(v) + \omega(u) + \omega(v)}.
    \]
    By the theorem of \S\,\ref{sec:hilbert-symbol-intro} we have
    \[
    (2u, 2v)_p = (2u, -4uv)_p = (2u, -uv)_p,
    \]
    and thus, we have
    \begin{equation}
    (2u, 2v)_p = (-1)^{\epsilon(u)\epsilon(-uv) + \omega(-uv)}
    \label{eq:2u-2v}
    \end{equation}
    Now since \(\epsilon(-1) = 1\), \(\omega(-1) = 0\), and \(\epsilon(u) (1 +
    \epsilon(u)) = 0\), the exponent in \eqref{eq:2u-2v} is equal to
    \(\epsilon(u)\epsilon(v) + \omega(u) + \omega(v)\). \scshape{q.e.d.}
\end{enumerate}




\subsection{Global properties.}\label{sec:hilbert-reciprocity}~The field
\(\Rationals\) embeds in \(\Rationals_v\) for each \(v \in \Places\). We want to
prove two results about the Hilbert symbol in \(\Rationals\) in this section.
The first is the product formula, which is a generalization of the law of
quadratic reciprocity. The second is a theorem that gives us a way to determine
whether rational numbers exists with given Hilbert symbols.

\begin{theoremx}
    If \(a, b \in \Rationals^{\times}\), then \((a, b)_v = 1\) for all but
    finitely many \(v \in \Places\) and 
    \[
        \prod_{v \in \Places} (a, b)_v = 1.
    \]
\end{theoremx}

\begin{proof}
    See \cite[p.~23]{weismann2006annotations}. Since the Hilbert symbol is
    bilinear, it suffices to prove the theorem for \(a\) and \(b\) equal to
    \(-1\) or a prime. We consider each case, making use of the identities in
    the preceding section.

    \smallskip

    \begin{enumerate}[nosep, label=(\roman*), wide]
        \item If \(a = b = -1\), then \((a, b)_v = (-1, -1)_v = -1\) for all \(v
        \in \Places\).
        \item If \(a = -1\) and \(b = \pi\), where \(\pi\) is a prime, then if
        \(\pi = 2\), we have \((a, b)_v = (-1, 2)_v = 1\) for all places \(v\).
        If \(\pi\) is odd, then \((a, b)_v = (-1, \pi)_v = 1\) for all \(v \neq
        2\) and \((a, b)_2 = (-1, \pi)_2 = (-1, \pi)_{\pi} =
        (-1)^{\epsilon(\pi)} = 1\).\label{item:case-ii-hilbert-prod}
        \item If \(a\) and \(b\) are the same prime \(\pi\), then by the theorem
        of \S\,\ref{sec:hilbert-symbol-intro}, \((\pi, \pi)_v = (-1, \pi)_v\)
        for all places \(v\) and we are back to case
        \ref{item:case-ii-hilbert-prod}.
        \item If \(a = 2\) and \(b\) is an odd prime \(\pi'\), then \((2,
        \pi')_v = 1\) for all \(v \notin \{2, \pi'\}\) while \((2, \pi')_2 =
        (-1)^{\omega(\pi')}\) and \((2, \pi')_{\pi'} = \legendre{2}{v}\), for
        \(v = 2\) and \(v = \pi'\), respectively; in either of these last two
        cases, we have \((2, \pi')_v = (-1)^{(\pi'^2 - 1)/8} = 1\).
        \item If \(a\) and \(b\) are distinct odd primes \(\pi\) and \(\pi'\),
        then \((a,b)_v = 1\) for all \(v \notin \{2, \pi, \pi'\}\). In those
        cases we have \((a, b)_v = (-1)^{\epsilon{\pi}\epsilon{\pi'}}\) if \(v =
        2\); \((a, b)_v = \legendre{\pi}{\pi'}\) if \(v = \pi\); and \((a, b)_v
        = \legendre{\pi'}{\pi}\) if \(v = \pi'\). The result follows by
        quadratic reciprocity, \emph{viz.,}
        \[
            \prod (a, b)_v =  (-1)^{\frac{\pi-1}{2}\frac{\pi'-1}{2}} \legendre{\pi}{\pi'} \legendre{\pi'}{\pi} = 1.
        \]
    \end{enumerate}
\end{proof}

\begin{theoremx}\label{thm:global-properties-hs} Let \(\{x_i\}_{i \in I}\) be a
    finite family of elements of \(\Rationals^{\times}\) and let
    \(\{\epsilon_{i, v}\}_{i \in I, v \in \Places}\) be a family of numbers
    equal to \(\pm 1\). There exists \(\lambda \in \fieldU\) such that \((x_i,
    \lambda)_v = \epsilon_{i, v}\) for all \(i \in I\) and \(v \in \Places\) if
    and only if the following conditions are satisfied:

    \smallskip

    \begin{enumerate}[nosep, label=(\alph*)]
        \item almost all the \(\epsilon_{i, v}\) are equal to \(1\);
        \item for all \(i \in I\), we have \(\prod_{v \in \Places} \epsilon_{i,
        v} = 1\); and
        \item for all \(v \in \Places\), there exists \(\lambda_v \in
        \Rationals_v^{\times}\) such that \((x_i, \lambda_v)_v = \epsilon_{i,
        v}\) for all \(i \in I\).\label{item:condition-c}
    \end{enumerate}
\end{theoremx}

% We again refer the reader to \cite[pp.~24--26]{serre2012course} for a proof.

We first introduce a couple of lemmas to prove
Theorem\,\ref{thm:global-properties-hs}.

\begin{lemmax}[``Chinese remainder theorem'']\label{lem:chinese-remainder} Let
    \(a_1, \dots, a_n\) and \(m_1, \dots, m_n\) be integers, with the \(m_i\)
    pairwise coprime. Then there exists an integer \(x\) such that \(x \equiv
    a_i \pmod{m_i}\) for all \(i\).
\end{lemmax}

\begin{proof}
    This result has been proved in class. Alternatively, let \(m\) be the
    product of the \(m_i\). By B\'ezout's theorem, the map
    \[\Integers/m\Integers \to \prod_{i = 1}^n \Integers/m_i\Integers\] is an
    isomorphism. The result follows from this fact.
\end{proof}

\begin{lemmax}[``Approximation lemma'']\label{lem:approximation}
    {\normalfont\cite[pp.~24--25]{serre2012course}} Let \(S \subset \Places\) be
    a finite set of places. The image of \(\Rationals\) is dense in \(\prod_{v
    \in S} \Rationals_v\).
\end{lemmax}
    

\emph{Proof.} Suppose that \(S = \{\infty, p_1, \dots, p_n\}\) with the \(p_i\)
distinct. We can assume this without loss of generality since if the image of
\(\Rationals\) is dense in \(\prod_{v \in S} \Rationals_v\) for any finite set
\(S\) of places, then it is also dense in \(\prod_{v \in S'} \Rationals_v\) for
any subset \(S' \subset S\). Thus we may enlarge \(S\) as needed and assume
\(S\) is as we have defined it.

Now fix an element \((x_{\infty}, x_1, \dots, x_n)\) of \(S\) so that
multiplying it by some integer, we can assume that the \(x_i\) are all elements
of \(\Integers_p\). Now let \(\epsilon > 0\) be any real number and \(N\) a
natural number. By the Chinese remainder theorem, there exists an integer
\(x_0\) such that \(x_0 \equiv x_i \pmod{p_i^N}\) for all \(i\). Therefore we
have
\[
    \nu_{p_i}(x_0 - x_i) \geq N.
\] 
Choose now an integer \(\pi \geq 2\) relatively prime to each of the \(p_i\)  so
that rational numbers of the form \(a/\pi^m\) are dense in \(\Rationals\) for
\(a \in \Integers\) and \(m \in \Naturals\). (This is possible since \(\pi^m\)
diverges as \(m\) tends to infinity.) We can thus choose a number \(\omega =
a/\pi^m\) such that
\[
    |x_0 - x_{\infty} + \omega p_1^N\cdots p_n^N| \leq \epsilon,
\]
so that setting \(x = x_0 + \omega p_1^N\cdots p_n^N\) we have
\begin{align*}
    |x - x_{\infty}| \leq \epsilon, \text{ and } \nu_{p_i}(x - x_i) \geq N,
\end{align*}
for all \(i\), as desired. {\scshape q.e.d.}

\medskip

\begin{lemmax}[``Dirichlet's theorem on arithmetic
    progressions'']\label{lem:dirichlet}
    {\normalfont\cite[p.~25]{serre2012course}} If \(a\) and \(m\) are relatively
    prime positive integers, then there exist infinitely many primes \(p\) such
    that \(p \equiv a \pmod{m}\).
\end{lemmax}

This lemma is proved by showing that the value of the Dirichlet \(L\)-function
at \(s = 1\) is nonzero. The proof requires some results from complex analysis
and analytic number theory that are beyond the scope of this paper. We refer the
reader to Chapter 6 of \cite{serre2012course} for the development and the proof
of this result.

We are now ready to prove Theorem\,\ref{thm:global-properties-hs}.

\emph{Proof of Theorem} \ref{thm:global-properties-hs}. Let
\(\{\epsilon_{i,v}\}\) a family of numbers equal to \(\pm 1\) satisfying the
conditions of the theorem. Multiplying the \(x_i\) by the square of some
integer, we can assume that the \(x_i\) are all integers (recall that the
Hilbert symbol is trivial on squares). Let now \(S\) be the set of all prime
factors of the \(x_i\) together with the symbol, \(\infty\) and the prime number
\(2\). In addition, define the set \(T\) be the places \(v \in \Places\) such
that there eixsts \(i \in I\) with \(\epsilon_{i, v} = -1\). The reader may
verify that these sets  are finite. We can now distinguish between two cases.

\begin{enumerate}[nosep, label=(\roman*), wide]
    \item In the first case we have \(S \cap T = \emptyset\). Let \(a\) be the
    product of all places \(\pi \in T\) and \(m\) the product of all odd places
    \(\pi \in S\) multiplied by \(8\), in both cases excepting the case where
    \(\pi = \infty\). Because \(S\) and \(T\) are disjoint, the integers \(a\)
    and \(m\) are relatively prime, and by Lemma\,\ref{lem:dirichlet} there
    exists a prime number \(p\) such that \(p \equiv a \pmod{m}\) with \(p
    \notin S \cup T\). We shall now show that \(\lambda = ap\) has our desired
    property, i.e., \((x_i, \lambda)_v\) for all \(i \in I\) and \(v \in
    \Places\). 
    
    If \(v \in S\), we have \(\epsilon_{i, v} = 1\) since \(S\) and \(T\) are
    disjoint, and we must check that \((x_i, \lambda) = 1\). If \(v = \infty\),
    this fact follows from \(\lambda > 0\); if \(v\) is a prime number \(\pi\),
    we have \(\lambda \equiv a^2 \pmod{m}\), hence \(\lambda \equiv a^2
    \pmod{8}\) for \(\pi = 2\) and \(\lambda \equiv a^2 \pmod{\pi}\) for \(\pi\)
    odd. Since \(\lambda\) and \(a\) are \(\pi\)-adic units, it follows that
    \(x\) is a square in \(\Rationals_{\pi}^{\times}\) and thus \((x_i,
    \lambda)_v = 1\).%TODO ADD REF

    Otherwise, if \(v = \pi\) is not in \(S\), \(x_i\) is a \(\pi\)-adic unit;
    and since \(\pi\) is an odd prime we have, by the theorem of
    \S\,\ref{sec:computing-hilbert-symbol},
    \[
        (x_i, \mu)_{\pi} = \legendre{x_i}{\pi}^{\nu_\pi(\mu)}
    \]
    for all \(\mu \in \Rationals_{\pi}^{\times}\). If \(\pi\) is not in \(T \cup
    \{p\}\), \(\lambda\) is a \(\pi\)-adic unit, hence the \(\pi\)-adic
    valuation \(\nu_{\pi}(\lambda) = 0\) and the above formula shows that
    \((x_i, \lambda) = 1\); on the other hand, we have \(\epsilon_{i, \pi} = 1\)
    because \(\pi \notin T\). If \(\pi\) is in \(T\), we then have
    \(\nu_{\pi}(\lambda) = 1\); moreover, by condition
    {\normalfont\ref{item:condition-c}} of the theorem, there exists a
    \(\lambda_{\pi} \in \Rationals_{\pi}^{\times}\) such that \((x_i,
    \lambda_{\pi})_{\pi} = \epsilon_{i, \pi}\) for all \(i \in I\). Since one of
    the \(\epsilon_{i, \pi}\) is equal to \(-1\) (because \(pi \in T\)), we have
    \(\nu_\pi(\lambda_\pi) = 1 \pmod{2}\) and therefore
    \[
        (x_i, \lambda)_{\pi} = \legendre{x_\pi}{\pi} = (x_i, \lambda_{\pi})_{\pi} = \epsilon_{i, \pi} \text{ for all } i \in I.
    \]

    Finally, we consider the case \(\pi = p\), which we deduced from the
    previous cases using the product formula, \emph{viz.,}
    \[
        (x_i, \lambda)_p = \prod_{v \neq p} (x_i, \lambda)_v = \prod_{v \neq p} \epsilon_{i, v} = \epsilon_{i, p}.
    \]
    This concludes the proof of the theorem in the case \(S \cap T =
    \emptyset\).

    \item Now we shall prove the general case. The squares of
    \(\Rationals_v^{\times}\) form an open subgroup of \(\Rationals_v^{\times}\)
    (cf. \cite[p.~18]{serre2012course}). By the approximation lemma above, there
    exists \(\lambda' \in \Rationals^{\times}\) such that \(\lambda' /
    \lambda_v\) is a square in \(\Rationals_v^{\times}\) for all \(v \in S\). In
    particular, \((x_i, \lambda')_v = (x_i, \lambda_v)_v = \epsilon_{i, v}\) for
    all \(v\) in \(S\). Define \(\eta_{i, v} = \epsilon_{i, v} (x_i,
    \lambda')_v\); the family \(\{\eta_{i, v}\}\) satisfies the conditions of
    the theorem and moreover, \(\eta_{i, v} = 1\) for all \(v \in S\). By the
    first case in this proof (\(S \cap T = \emptyset\)), there exists \(\zeta
    \in \Rationals^{\times}\) such that \((x_i, \zeta)_v = \eta_{i, v}\) for all
    \(i \in I\) and \(v \in \Places\). Setting \(\lambda = \lambda' \zeta\) we
    obtain the desired result. {\scshape q.e.d.}
\end{enumerate}

\section{Quadratic forms over \(\FiniteHead_q\) and \(\RationalsHead_p\)}

\subsection{Some results from Witt.}~In \S\,\ref{sec:mat-equiv} we have
established that the forms \(xy\) and \(x^2 - y^2\) are equivalent over
\(\Rationals\). As it turns out, these forms relate to more general forms over
\(\Rationals\) and over \(\Rationals_p\) and are thus particularly useful in
understanding quadratic forms over these fields, as has been demonstrated by the
German mathematician Ernst Witt in his seminal paper from 1937
\cite{witt1937theorie}. We refer the reader to \cite{lam1973quadratic} for a
more detailed treatment of Witt's decomposition and cancellation theorems, but
we shall state here the results we need.\label{sec:results-from-witt}

We say that an element \(x\) of the quadratic space \((V,B)\) is
\emph{isotropic} if it represents zero, i.e., if there exists a nonzero vector
\(x \in V\) such that \(q(x) = 0\), where \(q\) is the associated quadratic
form. We say that the whole space is isotropic if it contains an isotropic
vector. If two isotropic elements \(x\) and \(y\) of \((V,B)\) satisfying
\(B(x,y) \neq 0\) forms a basis of \(V\), then we call \((V, B)\) the hyperbolic
plane. If \((V,B)\) is the hyperbolic plane then \(B\) is isomorphic to the
orthogonal sum \(n\langle 1 \rangle \perp n \langle -1 \rangle\), where again
\(\langle \lambda \rangle\) denotes the quadratic space of rank \(1\) with
discriminant \(\lambda\), i.e., the isometry class of \(\lambda\). We can
summarize these results in the following theorem.

\begin{theoremx}\label{thm:regular-witt} Let \((V, B)\) be a quadratic space
    over \(\Rationals\) of rank \(2\). Then the following statements are
    equivalent:

    \smallskip

    \begin{enumerate}[nosep, label=(\alph*)]
        \item \((V, B)\) is regular and isotropic.
        \item \(B \cong \langle 1, -1 \rangle \cong \langle 1 \rangle \perp
        \langle -1 \rangle\).
        \item \(B\) is regular and \(\discr B \in -1 \cdot \SquareClass\).
        \item \(B\) is isomorphic to the equivalence class of the form \(xy\).
    \end{enumerate}
\end{theoremx}

\emph{Proof.} The results are immediate but cf.
\cite[pp.~12--13]{lam1973quadratic}.

\begin{theoremx}\label{thm:hyperbolic-decomp} Let \((V,B)\) be a regular
    quadratic space and let \(x \neq 0\) be an isotropic element of \(V\). Then
    there exists a subspace \(W\) of \(V\) such that \(W\) contains \(x\) and is
    hyperbolic.
\end{theoremx}

\begin{proof}
    See also \cite[p.~13]{clarkquadratic} or \cite[p.~29]{serre2012course}.
    Since \((V,B)\) is regular, there exists a vector \(z \in V\) such that
    \(B(x,z) \neq 0\); we can assume without loss of generality that \(B(x,z) =
    1\). We claim that there exists a \(\lambda \in \field\) such that
    \(q(\lambda x + z) = 0\) where \(q\) is the associated quadratic form.
    Indeed we have
    \[
        q(\lambda x + z) = \lambda^2q(x) + 2\lambda B(x,z) + q(z) = 2\lambda + q(z).     
    \]
    Let \(\lambda = -q(z)/2\). Then if \(y = \lambda x + z\), we have \(q(x) =
    q(y) = 0\) and
    \[
        B(x,y) = B(x, \lambda x + z) = \lambda q(x) + B(x,z) = 1.
    \]
    Thus the subspace spanned by \(x\) and \(y\) is hyperbolic.
\end{proof}

\medskip


We say that a quadratic space \((V,B)\) is \emph{universal} if it represents
every element of \(\fieldU\), i.e., \(D(B) = \fieldU\).  We can deduce the
following corollary.

\begin{corollaryx}\label{cor:regular-isotropic} Every regular isotropic
    quadratic space is universal.
\end{corollaryx}

The following corollary can also be deduced from the above theorem.

\begin{corollaryx}[First representation theorem]\label{cor:rep-theorem-1} Let
    \(q\) be a nondegenerate quadratic form and let \(\lambda \in \fieldU\).
    Then \(\lambda \in D(q)\) if and only if \(q \perp \langle -\lambda
    \rangle\) is isotropic.
\end{corollaryx}

\begin{proof}
    See \cite[pp.~14--15]{lam1973quadratic}. We can assume without loss of
    generality that \(q\) is a diagonal form. Now suppose \(\lambda \in D(q)\).
    Then \(\lambda = \sum \lambda_i x_i^2\) for some \(x_i \in \field\), so that
    \(\left(\sum \lambda_ix_i^2 + (-\lambda) \cdot 1^2 \right) = 0\) and hence
    \(q \perp \langle -\lambda \rangle\) is isotropic. Conversely, if \(q \perp
    \langle -\lambda \rangle\) is isotropic, then there exists a vector \(x =
    (x_1, \dots, x_{n+1})\) such that \(q(x) = 0\). If \(x_{n+1} \neq 0\), then
    \[
        \lambda = \sum \lambda_i\left(\frac{x_i}{x_{n+1}}\right)^2 \in D(q).
    \]
    Otherwise, if \(x_{n+1} = 0\), then \(x' = (x_1, \dots, x_n) \neq 0\) is an
    isotropic vector of \(q\), whence \(D(q) = \fieldU\) and \(\lambda \in
    D(q)\).
\end{proof}

We now ``translate'' Theorem \ref{thm:hyperbolic-decomp} and its corollaries in
the language of quadratic forms. To simplify things, we shall again abuse
notation and write \(q + r\) for the orthogonal sum of quadratic forms \(q\) and
\(r\) of rank \(n\) and \(m\) respectively, i.e.,
\[q + r = f(x_1, \dots, x_n) + g(x_{n+1}, \dots, x_{n+m}).\] Similarly, we put
\(q - r\) for \(q + (-r)\). Moreover, we say that \(q\) is hyperbolic if
\[
    q \sim xy \sim x^2 - y^2.
\]
The following theorem is therefore a restatement of Theorem
\ref{thm:hyperbolic-decomp} and Corollary \ref{cor:regular-isotropic} above:

\begin{theoremx}\label{thm:hyperbolic-decomp-2} Let \(q\) be a nondegenerate
    quadratic form representing zero. Then \(q\) is equivalent to the form \(q'
    + r\) where \(q'\) is hyperbolic and \(r\) is of rank \(n - 2\). Moreover,
    \(q\) is universal.
\end{theoremx}

\begin{corollary}
    Let \(q\) be a nondegenerate quadratic form in \(n - 1\) variables and let
    \(\lambda \in \fieldU\). Then the following statements are equivalent:

    \smallskip

    \begin{enumerate}[nosep, label=(\alph*)]
        \item The form \(q\) represents \(\lambda\).
        \item If \(r\) is a quadratic form in \(n - 2\) variables, then \(q \sim
        r + \lambda z^2\) for some variable \(z\).
        \item The form \(s = q - \lambda z^2\) represents zero.
    \end{enumerate}
\end{corollary}

\begin{proof}
    The equivalence (a) \(\iff\) (c) is Corollary \ref{cor:rep-theorem-1} (the
    first representation theorem) of Theorem \ref{thm:hyperbolic-decomp} above.
    The implication (b) \(\implies\) (a) is trivial. For the converse, suppose
    \(q\) represents \(\lambda\); then there exists a vector \(x\) such that
    \(q(x) = \lambda\). Let \(H\) be the orthogonal complement of \(x\) so that
    \(V = H \perp \langle \lambda \rangle.\) If \(r\) is the quadratic form
    associated with \(H\), then \(q \sim r + \lambda z^2\) for some variable
    \(z\).
\end{proof}

\subsection{Quadratic forms over \(\FiniteField_q\).}~We first look at quadratic
forms over a finite field \(\FiniteField_q\) with \(q\) elements where \(q\) is
a power of an odd prime \(p\). The two theorems in this section show us that
quadratic forms over \(\FiniteField_q\) are completely determined, up to
equivalence, by their rank and discriminant. The results in this section are not
needed in the sequel.\label{sec:quadratic-forms-fq}

\begin{theoremx}{\normalfont\cite[p.~34]{serre2012course}} A quadratic form over
    \(\FiniteField_q\) of rank \(r \geq 2\) (resp. \(r \geq 3\)) is universal in
    \(\FiniteField_q^\times\) (resp.
    \(\FiniteField_q\)).\label{thm:universal-forms-in-fq}
\end{theoremx}

\begin{proof}
    Following the corollary of Theorem \ref{thm:hyperbolic-decomp-2} of the
    preceding section, it suffices to show that all quadratic forms in \(3\)
    variables represent \(0\). This fact is a consequence of the
    Chevalley-Warning theorem, which we have not taken up in this paper. See
    also the proof of Lemma \ref{lem:hilbert-odd-units} in
    \S\,\ref{sec:computing-hilbert-symbol}. 
\end{proof}

\begin{theoremx}\label{thm:quadratic-forms-fq-rank-n} Every nondegenerate form
    over \(\FiniteField_q\) of rank \(n\) is equivalent to the form
    \[
        x_1^2 + \cdots + x_{n-1}^2 + x_n^2.
    \]
    or to the form
    \[
        x_1^2 + \cdots + x_{n-1}^2 + \lambda x_n^2
    \]
    where \(\lambda\) is a nonsquare element of \(\FiniteField_q\), depending on
    whether its discriminant is a square or not.
\end{theoremx}

\begin{proof}
    We proceed by induction. Let \(q\) be a quadratic form of rank \(n\) in
    \(\FiniteField_q\). For \(n = 1\), every quadratic form is of the form
    \(x^2\) or \(ax^2\) since the group \(\FiniteField_q^{\times} /
    \FiniteField_q^{\times2}\) has order \(2\). If \(n \geq 2\) then by Theorem
    \ref{thm:universal-forms-in-fq} \(q\), being universal, represents \(1\);
    thus \(q \sim x^2 + r\) for some quadratic form \(r\) of rank \(n - 1\) and
    by induction, we are done.
\end{proof}

\subsection{The Hasse-Minkowski invariant.}~Let \(\field\) be \(\Rationals_p\)
or \(\Reals\) (we shall maintain this assumption until the end of the next
section). Let \((V, B)\) be a quadratic space over \(\field\) with \(q\) as its
associated quadratic form and let \(\discr q\) be its discriminant. We have
established in \S\,\ref{sec:square-classes} that \(\discr B\) is an element of
the square class \(\SquareClass\). Now if \(\Basis = \{b_1, \dots, b_n\}\) is an
orthogonal basis of \(V\) and we write \(a_{i} = B(b_i, b_i)\), then we
have\label{sec:hasse-invariant}
\[
    \discr q = a_1 \cdots a_n.  
\]
Recall now that we have defined the Hilbert symbol \((a, b)\) for elements \(a,
b \in \fieldU\). We then define
\[
    \epsilon(\Basis) = \prod_{i < j} (a_i, a_j)
\]
so that \(\epsilon := \epsilon(\Basis) = \pm 1\). We now have the following
theorem.

\begin{theorem}{\normalfont\cite[pp.~35--36]{serre2012course}} The number
    \(\epsilon\) does not depend on the choice of the orthogonal basis
    \(\Basis\) of \(V\).
\end{theorem}

\begin{proof}
    We prove the theorem by induction on the rank of \(V\). If \(n = 1\), we
    have \(\epsilon = 1\) as the product is trivial. If \(n = 2\), then
    \(\epsilon = 1\) if and only if \(z^2 - ax^2 - by^2\) has a nontrivial
    solution in \(\field^3\), which is true, by the corollary of Theorem
    \ref{thm:hyperbolic-decomp-2} of the preceding section, if and only if
    \(ax^2 + by^2\) represents \(1\). But this last statement implies that there
    exists a vector \(v \in V\) such that \(q(x) = 1\); this vector exists
    independent of the choice of basis.

    For \(n \geq 3\), we proceed by induction on \(n\). By the theorem of
    \S\,\ref{sec:contiguous-bases}, it suffices to show that \(\epsilon(\Basis)
    = \epsilon(\Basis')\) whenever \(\Basis\) and \(\Basis'\) are contiguous
    bases of \(V\). Moreover, because the Hilbert symbol is symmetric,
    \(\epsilon(\Basis)\) does not change no matter how we permute the elements
    of \(\Basis\). Thus we can assume without loss of generality that \(b_1 =
    b'_1\) and hence \(a_1 = a'_1\). We then have
    \begin{align*}
        \epsilon(\Basis)  &= (a_1, a_2 \cdots a_n) \prod_{2 \leq i <j} (a_i, a_j) \\
                                &= (a_1, \discr B a_1) \prod_{2 \leq i <j} (a_i, a_j)
    \end{align*}
    since \(\discr B = a_1 \cdots a_n\). Similarly, we have
    \[
        \epsilon(\mathfrak{B'}) = (a_1, \discr B a_1) \prod_{2 \leq i <j} (a'_i, a'_j).
    \]
    But applying the inductive hypothesis to the orthogonal complement of
    \(b_1\) gives us
    \[
        \prod_{2 \leq i <j} (a_i, a_j) = \prod_{2 \leq i <j} (a'_i, a'_j),  
    \]
    which completes the proof.
\end{proof}

In view of the above theorem we shall call \(\epsilon\) the
\emph{Hasse-Minkowski invariant}\index{Hasse-Minkowski invariant} of the
quadratic space \((V, B)\). We shall also sometimes write \(\epsilon(B)\) or
\(\epsilon(q)\) (where \(q\) is the form associated with \(B\)) to emphasize the
dependence of \(\epsilon\) on \(B\) or \(q\).\footnote{The notation \(c_p(q)\)
for the Hasse-Minkowski invariant of the form \(q\) also appears in the
literature, as in, e.g., \cite{jones1950arithmetic,cassels2008rational}.}

\subsection{Representation in \(\Rationals_p\).}~In this section we provide the
necessary and sufficient conditions for any form over \(\Rationals_p\) to
represent zero.\label{sec:representation-in-qp-sec}

\begin{theoremx}\label{thm:rep-in-rationals-p} Let \(q\) be a nondegenerate
    quadratic form over \(\Rationals_p\) of rank \(n\) and let \(\discr q\) and
    \(\epsilon\) be its discriminant and Hasse-Minkowski invariant respectively.
    Then for \(q\) to represent \(0\) in \(\Rationals_p\) it is necessary and
    sufficient that:

    \smallskip

    \begin{enumerate}[nosep, label=(\alph*)]
        \item \(n = 2\) and \(\discr q = -1\) (in \(\SquareClass\));
        \item \(n = 3\) and \((-1, \discr q) = \epsilon\);
        \item \(n = 4\), \(\discr q = \pm 1\) and \(\epsilon = (-1, -1)\); or
        \item \(n \geq 5\).
    \end{enumerate}
    That is, all forms of rank \(n \geq 5\) represent \(0\) in \(\Rationals_p\).
\end{theoremx}

We shall not prove this theorem in this paper but the reader may refer to
\cite[pp.~36--39]{serre2012course} for a proof. Now note that if \(\lambda \in
\SquareClass\) and \(q_{\lambda}\) is the form \(q - \lambda z^2\), then we know
by the results in \S\,\ref{sec:results-from-witt} that \(q_{\lambda}\)
represents \(0\) in \(\Rationals_p\) if and only if \(q\) represents
\(\lambda\). Noting also the fact that \(\discr q_{\lambda} = -\lambda \discr
q\) (cf. Theorem \ref{thm:regular-witt} of \S\,\ref{sec:results-from-witt}) and
\(\epsilon(q_\lambda) = (-\lambda, \discr q)\epsilon\), we can deduce the
following corollary, which provides necessary and sufficient conditions for a
form \(q\) to represent any \(\lambda \in \SquareClass\), not just \(0\).

\begin{corollary}
    Let \(\lambda \in \SquareClass\) and let \(q\), \(\discr q\) and
    \(\epsilon\) be as above. Then for \(q\) to represent \(\lambda\) in
    \(\Rationals_p\) it is necessary and sufficient that:

    \smallskip

    \begin{enumerate}[nosep, label=(\alph*)]
        \item \(n = 1\) and \(\lambda = \discr q\);
        \item \(n = 2\) and \((\lambda, -\discr q) = \epsilon\);
        \item \(n = 3\) and \(\lambda = \pm \discr q\) and \(\epsilon = (-1,
        \discr q)\); or
        \item \(n \geq 4\).
    \end{enumerate}
\end{corollary}

With these results, we are now able to classify, up to equivalence, all
quadratic forms over \(\Rationals_p\). We summarize the results in the following
theorem.

\begin{theoremx}{\normalfont\cite[p.~39]{serre2012course}} Two quadratic forms
    over \(\field = \Rationals_p\) or \(\Reals\) are equivalent if and only if
    they have the same rank, discriminant and Hasse-Minkowski invariant.
\end{theoremx}

\begin{proof}
    The forward implication follows directly from the definition of equivalence.
    We shall prove the converse by induction on the rank \(n\) of two  forms
    \(q\) and \(r\). We also write \(\Delta\) and \(\epsilon\) for the
    discriminant \(\Delta q = \Delta r\) and the Hasse-Minkowski invariant
    \(\epsilon(q) = \epsilon(r)\). The case \(n = 0\) is trivial. By the
    corollary of Theorem\,\ref{thm:rep-in-rationals-p} of this section, \(q\)
    and \(r\) represent the same element \(\lambda \in \fieldU\). We then have
    \[
        q \sim \langle \lambda \rangle \perp q' \quad \text{and} \quad r \sim \langle \lambda \rangle \perp r'
    \]
    where \(q'\) and \(r'\) are quadratic forms of rank \(n - 1\). We then have
    \[
      \discr q' = \lambda \discr q = \lambda \discr r = \discr r'
    \]
    and 
    \[
        \epsilon(q') = (-\lambda, \discr q')\epsilon = (-\lambda, \discr r')\epsilon = \epsilon(r'),
    \]
    and thus \(q'\) and \(r'\) have the same rank, discriminant and
    Hasse-Minkowski invariant. By the inductive hypothesis, \(q'\) and \(r'\)
    are equivalent, and hence \(q\) and \(r\) are equivalent.
\end{proof}

We conclude this section by exploring in a bit more detail quadratic forms over
the field \(\Reals\) of real numbers. These results are again from
\cite{serre2012course}, but shall not be needed in the sequel. We know from
\S\,\ref{sec:results-from-witt} that if \(q\) is a quadratic form over
\(\Reals\) of rank \(n\), then \(q\) is equivalent to the form
\[
    x_1^2 + \cdots + x_{r}^2 - y_1^2 - \cdots - y_{s}^2
\]
where \(r\) and \(s\) are positive integers satisfying \(r + s = n\). We shall
now show that the integers \(r\) and \(s\) depend only on the form \(q\) and is
thus an invariant of \(q\) over \(\Reals\). We shall call the pair \((r,s)\) the
\emph{signature}\index{signature} of \(q\). We say that \(q\) is \emph{positive
definite} if \(r = n\) and \emph{negative definite} if \(s = n\). We say that
\(q\) is \emph{indefinite} if \(r\) and \(s\) are both positive. If we let
\(\epsilon\) and \(\discr q\) be the Hasse-Minkowski invariant and discriminant
of \(q\) respectively, then, because \((-1, -1) = -1\),
\[
    \epsilon = (-1)^{s(s-1)/2} \quad \text{and} \quad \discr q = (-1)^{s}.
\]
This result is Sylvester's law of inertia from linear algebra. See also
\cite[pp.~83--84]{szymiczek2017bilinear}.

\section{The Hasse-Minkowski theorem.}

\subsection{The Hasse-Minkowski theorem.}~We are now ready to prove the main
result of this chapter, the Hasse-Minkowski theorem. As in
\S\,\ref{sec:hilbert-reciprocity}, we shall write \(\mathfrak{V}\) for the union
of the set of prime numbers and the symbol \(\infty\), so that writing
\(\Rationals_v\) for the \(v\)-adic field is well-defined for all \(v \in
\mathfrak{V}\), with \(\Rationals_{\infty} = \Reals\) by definition. As before,
we shall write \(\quadform{a_1, \dots, a_n}\) for the quadratic form \(q \sim
a_1x_1^2 + \cdots + a_nx_n^2\) over \(\Rationals\); we then associate with
\(q\):\label{sec:hasse-minkowski} 

\smallskip

\begin{enumerate}[nosep, label=(\alph*)]
    \item a discriminant \(\Delta q\) in \(\Rationals^{\times} /
    \Rationals^{\times 2}\) equal to \(a_1 \cdots a_n\);
    \item for any \(v \in \Places\) we define the injection \(q_v : \Rationals
    \hookrightarrow \Rationals_v\) so that \(q_v\) denotes the quadratic form
    \(q\) over \(\Rationals_v\), such that \(\discr_v q\) is the image of
    \(\discr q\) by \(\Rationals^{\times} / \Rationals^{\times 2} \to
    \Rationals_v^{\times} / \Rationals_v^{\times2}\) and
    \[
      \epsilon_v (q) = \prod_{i < j} (a_i, a_j)_v,  
    \]
    with \(\prod \epsilon_v(q) = 1\) over all the places \(v\); and
    \item a signature \((r,s)\) as defined in the preceding section.
\end{enumerate}

The numbers \(\discr_v q\), \(\epsilon_v(q)\) and \((r, s)\) all called the
\emph{local invariants} of \(q\) with respect to the place \(v\). With our
notation fixed, we can now state the Hasse-Minkowski theorem.

\begin{theoremx}\label{thm:hasse-minkowski} In order that a quadratic form over
    \(\Rationals\) represent zero, it is necessary and sufficient that for every
    place \(v\) of \(\Rationals\) the form \(q_v\) represents \(0\).
\end{theoremx}

Before proceeding with the proof, recall that in
\S\,\ref{sec:polynomials-over-zp} we have remarked that, while the polynomial
\(x^2 - 2\) has a solution in \(\Rationals_7\), we know for a fact that it does
not have a solution in \(\Rationals\). We see that this is the case because, as
the Hasse-Minkowski theorem tells us, a quadratic form has a solution in
\(\Rationals\) if and only if it has a solution in \(\Rationals_v\) for all \(v
\in \Places\). Indeed we see by routine verification that \(x^2 - 2\) has no
solutions in \(\Rationals_3\).


\emph{Proof.} The necessity is clear since \(\Rationals \subset \Rationals_p\)
for all \(p\) so we are going to prove sufficiency. Write the form \(q\) as its
equivalent diagonal form \(\quadform{a_1, \dots, a_n}\) where each \(a_i \in
\fieldU\). Replacing \(q\) with \(a_1 q\), we can assume, without loss of
generality, that \(a_1 = 1\). We shall then again follow
\cite[pp.~41--44]{serre2012course} and \cite[pp.~102--104]{gerstein2008basic}
and prove the theorem for \(n = 2, 3, 4\) and \(n \geq 5.\)

\begin{enumerate}[wide, nosep, label=(\alph*)]
    \item \textbf{Case 1,} \(n = 2\). Suppose \(q \cong \quadform{1, a}\). Then
    for \(q\) to represent \(0\) in \(\Rationals\), \(q\) must represent \(0\)
    in all the places \(v\) of \(\Rationals\); in particular, \(q\) must
    represent \(0\) in \(\Rationals_{\infty} = \Reals\). It therefore follows
    that \(a < 0\) and thus we only need to consider the form \(q \cong
    \quadform{1, -a}\), with \(0 < a \in \Rationals\), so that \(q\) represents
    \(0\) in \(\Rationals\) if and only if it is a square in \(\Rationals\).
    Since \(q\) represents \(0\) in all \(\Rationals_p\),  \(a\) is a square in
    \(\Rationals_p\). Thus writing \(a\) as the product \(\prod_p
    p^{\nu_p(a)}\), we have that \(\nu_p(a)\) is even and therefore, \(a\) is a
    square in \(\Rationals\) and hence \(q\) represents \(0\) in \(\Rationals\).
    
    \item \textbf{Case 2,} \(n = 3\). We can assume, without loss of generality,
    that \(q \cong \quadform{1, -a, -b}\) where \(a\) and \(b\) are squarefree
    integers (i.e. \(\nu_p(a)\) and \(\nu_p(b)\) are equal to \(0\) or \(1\) for
    all prime \(p\)) and \(|a| \leq |b|\). We shall prove this by induction on
    the integer \(m = |a| + |b|\). If \(m = 2\), then \(q \cong \quadform{1, \pm
    1, \pm 1}\), but since \(q\) is represents 0 in \(\Rationals_\infty =
    \Reals\), then \(q \not\cong \quadform{1, 1, 1}\) and hence \(q \cong
    \quadform{1, -1, -1}\).
    
    Now if \(m > 2\). Recall that we are assuming \(|a| \leq |b|\) so that \(|b|
    \geq 2\) and write \(b = \pm p_1 \cdots p_r\) where \(p_i\) are distinct
    primes (noting also our assumption that \(b\) is squarefree). We then use
    the following lemma:

    \begin{lemma}
        Let \(a\) and \(b\) be as above and let \(p\) be a prime number. Then
        {\normalfont (i)} \(a\) is a square modulo \(p\); and {\normalfont (ii)}
        \(a\) is a square modulo \(b\).
    \end{lemma}

    We shall not prove this lemma but see \cite[p.~42]{serre2012course} for a
    proof. Here (ii) is a corollary of (i). Now since \(a\) is a square modulo
    \(b\), there exists integers \(t\) and \(b'\) such that \(t^2 = a + b'b\)
    such that \(0 \leq |t| \leq |b|/2\). We then have
    \[ b'b = N(t + \sqrt{a})\] where \(N: \field(\sqrt{a})/\field \to \field\)
    is the norm mapping of the extension \(\field(\sqrt{a})/\field\) of
    \(\field\) where \(k\) is \(\Rationals\) or \(\Rationals_v\) (cf.
    \cite[p.940]{rotman2010advanced}, \cite[p.~289]{hungerford2012algebra}). We
    can then conclude that \(q\) represents \(0\) in \(\field\) if and only if
    the form \(q' \cong \quadform{1, -a, -b'}\) does. And since \(|b| \geq 2\),
    it follows that
    \[
        |b'| = \left|\frac{t^2-a}{b}\right| \leq \frac{|b|}{4} + 1 \leq |b|
    \]
    Again write \(b'\) as \(b''u^2\) for some squarefree integer \(b''\) and
    some integer \(u\). It follows \emph{a fortiori} that \(|b''| < |b|\) and
    thus the form \(q'' \cong \quadform{1, -a, -b''}\) represents \(0\) in
    \(\Rationals\) by the inductive hypothesis, whence \(q\) represents \(0\) in
    \(\Rationals\), as desired.

    \item \textbf{Case 3,} \(n = 4\). Let \(q \cong \quadform{a, b, -c, -d}\).
    By Theorem\,\ref{thm:hyperbolic-decomp-2}, it suffices to show that there
    exists some \(\lambda \in \fieldU\) (where again \(\field\) is either
    \(\Rationals\) or \(\Rationals_v\)) represented by both \(\quadform{a, b}\)
    and \(\quadform{-c, -d}\). Suppose \(\quadform{a, b, -c, -d}\) represents
    \(0\) in each \(\Rationals_v\); then \(\quadform{a, b}\) and \(\quadform{c,
    d}\) represent the same element, say \(\lambda_v\), in \(\Rationals_v\).
    Over \(\Rationals_v\), it follows that \(\quadform{a, b}\) represents
    \(\lambda_v\) if and only if \(\quadform{a, b, -\lambda_v}\) represents
    \(0\) if and only if \(\quadform{1, ab, -a\lambda_v}\) is isotropic. This is
    true if and only if \((-ab, a\lambda_v)_v = 1\), which is true if and only
    if \((a, b)_v = (-ab, \lambda_v)_v = 1\). Similarly, we have \((c,d)_v =
    (-cd, \lambda_v)_v = 1\).
    
    Using the product formula of \S\,\ref{sec:hilbert-reciprocity}, we have 
    \[
        \prod_{v \in \Places} (a, b)_v = \prod_{v \in \Places} (c, d)_v = 1,
    \]
    and thus by Theorem\,\ref{thm:global-properties-hs} of
    \S\,\ref{sec:hilbert-reciprocity} we can find a \(\lambda \in
    \Rationals^{\times}\) satisfying
    \[
        (a, b)_v = (-ab, \lambda)_v \quad \text{and} \quad (c, d)_v = (-cd, \lambda)_v,
    \]
    and thus the forms \(\quadform{a, b, \lambda}\) and \(\quadform{c, d,
    \lambda}\) are isotropic in each \(\Rationals_v\). By our proof in the case
    \(n = 3\) above, it follows that \(\quadform{a, b, \lambda}\) and
    \(\quadform{c, d, \lambda}\) are isotropic in \(\Rationals\) and thus both
    \(\quadform{a,b}\) and \(\quadform{c,d}\) represent \(\lambda\) in
    \(\Rationals\). Thus \(\quadform{a, b, -c, -d}\) represents \(0\) in
    \(\Rationals\) and the result follows.\label{hm-case:4}
    
    \item \textbf{Case 4,} \(n \geq 5\). Let \(q = \quadform{a_1, \dots, a_n}\)
    be a quadratic form of rank \(n \geq 5\). Let \(r = \quadform{a_1, a_2}\)
    and \(s = \quadform{-a_3, \dots, -a_n}\), so that \(q \cong r \perp s\),
    i.e., \(q = r - s\) by the notation we introduced in
    \S\,\ref{sec:results-from-witt}. Suppose that \(q\) represents zero over
    every \(\Rationals_v\). As in Case \ref{hm-case:4} above, ther exists
    \(\lambda_v \in \Rationals_v^{\times}\) such that \(r\) and \(s\) represent
    \(\lambda_v\) in \(\Rationals_v\). 
\end{enumerate}

\medskip

We can deduce the following corollary, which is a straightforward application of
the above theorem to the form \(\quadform{\lambda} - q\).

\begin{corollary}
    % \label{cor:hasse-minkowski-representation}
    For a quadratic form \(q\) to represent \(\lambda \in \Rationals^{\times}\),
    it is necessary and sufficient that the form \(q_v\) represents \(\lambda\)
    in \(\Rationals_v\) for all \(v \in \Places\).
\end{corollary}

The next theorem relates the Hasse-Minkowski theorem with our notion of
equivalence.\footnote{One may sometimes find
Theorem\,\ref{thm:hasse-minkowski-equivalence}, and not
Theorem\,\ref{thm:hasse-minkowski}, referred to as the Hasse-Minkowski theorem
in the literature.}

\begin{theoremx}\label{thm:hasse-minkowski-equivalence} Two quadratic forms over
    \(\Rationals\) are equivalent if and only if they are equivalent over
    \(\Rationals_v\) for all \(v \in \Places\).
\end{theoremx}

\begin{proof}
    Let \(q\) and \(r\) be equivalent quadratic forms over \(\Rationals\).
    Necessity follows directly from the definitions. We prove sufficiency by
    induction on the rank \(n\) of \(q\) and \(r\). The case \(n = 0\) is
    trivial. We thus suppose \(n > 0\), and by the corollary of
    Theorem\,\ref{thm:hasse-minkowski} above, we can assume that \(q\) and \(r\)
    represent the same element \(\lambda \in \Rationals^{\times}\). By the
    representation criterion of \S\,\ref{sec:diagonalization-of-forms}, \(q \sim
    \quadform{\lambda} \perp q'\) and \(r \sim \quadform{\lambda} \perp r'\)
    where \(q'\) and \(r'\) are quadratic forms of rank \(n - 1\) over
    \(\Rationals\). This implies \(r \sim r'\) over \(\Rationals_v\) since \(q
    \sim q'\) and \(\quadform{\lambda} \sim \quadform{\lambda}\) for every
    \(\Rationals_v\). From our induction hypothesis \(r \sim r'\) over
    \(\Rationals\) and hence, similarly, \(q \sim q'\) over \(\Rationals\).
\end{proof}

\section{Appendix: some classical results}

\subsection{Sum of squares.}~We now begin to move from the rational theory of
quadratic forms towards an initial result involving integral results, which we
shall have more time to study in detail in the next chapter. We now use the
results of this chapter to prove some classical results, due to Gauss, Lagrange
and Legendre, on the representation of integers as sums of squares. We follow
the treatment in \cite[pp.~45--47]{serre2012course} and
\cite[pp.~109--111]{gerstein2008basic}. We begin with a couple of lemmas.

\begin{lemmax}\label{lem:sum-of-three-squares-q2} {\normalfont
    \cite[pp.\,45--46]{serre2012course}} If \(\lambda \in \Rationals^{\times}\),
    then \(\lambda\) is a sum of three rational squares if and only if \(\lambda
    > 0\) and \(-\lambda\) is not a square in \(\Rationals_2\).
\end{lemmax}

\begin{proof}
    By the corollary to the Hasse-Minkowski theorem, we need to show that
    \(\lambda\) is represented in \(\Rationals_{\infty} = \Reals\) and in
    \(\Rationals_p\) for all prime \(p\). The case of \(\Reals\) gives the
    positivity condition. On the other hand, for each \(p\) the local invariants
    \(\discr_p\) and \(\epsilon_p\) are both equal to \(1\). If \(p \neq 2\),
    then
    \[
        (-1, -\discr_p)_p = (-1, -1)_p = 1 = \epsilon_p
    \]
    and thus by the corollary of Theorem\,\ref{thm:rep-in-rationals-p} of
    \S\,\ref{sec:representation-in-qp-sec}, \(\lambda\) is represented in
    \(\Rationals_p\). If \(p = 2\), we have
    \[
        (-1, -\discr_2)_2 = (-1, -1)_2 = -1 \neq 1 = \epsilon_2.
    \]
    Thus by the same corollary, \(\lambda\) is represented in \(\Rationals_2\)
    if and only if \(\lambda\) is different from \(-1\) in \(\Rationals^*_2 /
    \Rationals^{*2}_2\), i.e., if and only if \(-\lambda\) is not a square in
    \(\Rationals_2\).
\end{proof}

\begin{lemmax}[Davenport-Cassels]\label{lem:sum-of-three-squares-davenport-cassels}
    {\normalfont \cite[p.\,46]{serre2012course}} Let \(q\) be a quadratic form
    over \(\Rationals\) of rank \(n\) with integer coefficients, and let \(A =
    (a_{ij})\) be its associated matrix. Suppose further that for every \(x =
    (x_1, \dots, x_n) \in \Rationals^n \setminus \Integers^n\) there exists some
    \(y = (y_1, \dots, y_n) \in \Integers^n\) such that \(q(x - y) < 1\). Then
    if \(q\) represents \(\lambda\) in \(\Rationals\), \(q\) also represents
    \(\lambda\) in \(\Integers\).
\end{lemmax}

\begin{proof}
    We follow the proof in \cite[p.\,2]{clark2011representation}, which itself
    follows the proof by Serre with some variation (cf. p.\,3, \emph{loc.
    cit.}). For all \(x, y \in \Rationals^n\) define
    \[
        x \cdot y = \frac{1}{2} \left( q(x + y) - q(x) - q(y) \right).
    \]
    Then the map \((x, y) \mapsto x \cdot y\) is a bilinear form and \(x \cdot x
    = q(x)\). Note that for all \(x, y \in \Integers^n\), it does not
    necessarily follow that \(x \cdot y \in \Integers^n\) but \(2(x \cdot y) \in
    \Integers^n\).

    Let \(\lambda\) be an integer and suppose that there exists some \(x \in
    \Integers^n\) such that \(q(x) = \lambda\). Equivalently, there exists some
    \(t \in \Integers\) and \(x' \in \Integers^n\) such that \(q(x') = t^2
    \lambda\). Choose \(x'\) and \(t\) such that \(|t|\) is minimal, so that we
    need only show that \(|t| = 1\).
    
    By the hypothesis on \(x = x'/\lambda\), there exists an integer vector
    \(y\) such that \(q(x - y) < 1\). Now write
    \begin{align*}
        a &= y \cdot y - \lambda \qquad &b &= 2(\lambda t - x \cdot y) \\
        \tau &= at + b\qquad &\xi &= ax' + by.
    \end{align*}
    Then \(a\), \(b\), and \(\tau\) are integers and \(\xi\) an integer vector.
    We now prove two claims: first, that \(q(\xi) = \tau^2\lambda\) and second,
    that \(\tau = tq(x - y)\).
    
    For the first claim we have
    \begin{align*}
        q(\xi) &= \xi \cdot \xi = a^2 (x' \cdot x') + ab(2x' \cdot y) + b^2(y \cdot y) \\
               &= a^2 t^2 \lambda + ab(2\lambda t - b) + b^2(\lambda + a) \\
               &= \lambda (a^2 t^2 + 2abt + b^2) = \lambda (at + b)^2 = \tau^2 \lambda.
    \end{align*}
    Similarly, for the second claim we have
    \begin{align*}
        t\tau &= t(at + b) = at^2 + bt = t^2 (y \cdot y) - \lambda t^2 + 2\lambda t^2 - t(2x' \cdot y)\\
        &= t^2(y \cdot y) - t(2x' \cdot y) + x' \cdot x' = (ty-x') \cdot (ty-x') = t^2 q(x - y),
    \end{align*}
    and thus \(\tau = tq(x - y)\). Since \(|q(x - y)| < 1\), we have \(|\tau| <
    |t|\), condradicting the minimality of \(|t|\).
\end{proof}

We now arrive to the main result of this section, which is due to Gauss and
Legendre.

\begin{theorem}
    {\normalfont \cite[pp.\,46--47]{serre2012course}} A positive integer
    \(\lambda\) is a sum of three squares if and only if \(\lambda\) is not of
    the form \(4^\nu k\) for some integer \(\nu\), where \(k \equiv 7
    \pmod{8}\).
\end{theorem}

\begin{proof}
    We may suppose without loss of generality that \(\lambda\) is nonzero. Then
    the condition that \(\lambda\) be of the form \(4^\nu k\) where \(k \equiv 7
    \pmod{8}\) is equivalent to saying that \(-\lambda\) is not a square in
    \(\Rationals_2\). By Lemma\,\ref{lem:sum-of-three-squares-q2}, this implies
    that \(\lambda\) is represented by a sum of three squares in \(\Rationals\).
    Let \(q = x_1^2 + x_2^2 + x_3^2\) be this form. By
    Lemma\,\ref{lem:sum-of-three-squares-davenport-cassels}, we need to check
    that the form \(q\) satisfies the condition that for every rational vector
    \(x = (x_1, x_2, x_3)\) there exists some integer vector \(y = (y_1, y_2,
    y_3)\) such that \(q(x - y) < 1\). Now if \(x\) is a rational vector, we can
    choose \(y\) satisfying \(|y_i - x_i| \leq 1/2\) for all \(i\). Then
    \[
        \sum(x_i - y_i)^2 \leq \sum \left(\frac{1}{2}\right)^2 = \frac{3}{4} < 1,
    \]
    as desired. Therefore \(q\) represents \(\lambda\) in \(\Integers\).
\end{proof}

We can deduce the following corollary, due to Lagrange and commonly known as the
four-square theorem, as mentioned in the Preface.

\begin{corollary}
    Every positive integer is a sum of four squares.
\end{corollary}

\begin{proof}
    Let \(\lambda > 0\) be an integer. Write \(\lambda\) in the form
    \(4^{\nu}m\) where \(m\) is an integer not divisible by \(4\). If \(m \equiv
    1, 2, 3, 5, 6 \pmod{8}\) then \(m\) is a sum of three squares and so is
    \(\lambda\). Otherwise if \(m \equiv 7 \equiv -1 \pmod{8}\) then \(m-1\) is
    the sum of three squares. Hence \(m\) is the sum of four squares and so is
    \(\lambda\).
\end{proof}

