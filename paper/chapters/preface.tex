\chapter*{Preface}

{\scshape The aim} of this paper is to provide a self-contained exposition of the proof of the Conway-Schneeberger Fifteen Theorem provided by Manjul Bhargava in \cite{bhargava2000conway}. The theorem states that for a positive-definite integer-valued quadratic form to represent all positive integers, it is necessary and sufficient that it represent all positive integers up to \(15\). The original proof by Conway was never published; in 1995, William Schneeberger has provided a sketch of the proof in his Ph.D. thesis. \cite{schneeberger1997arithmetic} The proof we examine here was provided by Bhargava in 2000, relying heavily on the theory of genera and of lattices; Bhargava was later to generalize this in what would later be known as the \(290\) theorem. \cite{bhargava2005universal} For this and his other contributions to number theory, in particular to the geometry of numbers, Bhargava was awarded the Fields Medal in 2014. \cite{bhargava2014fields}

While the author has endeavoured to make this exposition as self-contained as possible, some elementary results have been omitted and throughout the paper, the reader will be assumed to have a basic understanding of number theory, linear algebra, and abstract algebra, on the level of the introductory graduate courses on those subjects in De La Salle University. We refer the reader to \cite{dudley1978elementary,halmos1942vector,hungerford2012algebra} for a review of the necessary background.

Chapter \ref{chap:quadratic-forms} provides an introduction of the theory of quadratic forms over a field \(\field\) of characteristic not equal to \(2\) and the equivalent notions of Gram matrices and symmetric bilinear forms, following the treatment in \cite{lam1973quadratic}. The main results in the chapter include the equivalence of quadratic forms to a diagonal form as well as some results specific to the number theoretic treatment of rational and integral quadratic forms from \cite{cassels2008rational} and \cite{jones1950arithmetic}. Chapter \ref{chap:local-global-principle} is essentially an exposition of Chapters 2--4 of Serre's \emph{Course in Arithmetic} \cite{serre2012course}. In it, we construct the field \(\Rationals_p\) of \(p\)-adic numbers, develop the theory of quadratic forms in \(\Rationals_p\) and \(\Rationals\), and finally prove the Hasse-Minkowski theorem. We then explore some consequences of this theorem and its applications to number theory. In Chapter \ref{chap:integral-quadratic-forms} we introduce the language of lattices to study quadratic forms over the the rings \(\Integers\) and \(\Integers_p\) in order to investigate how, if at all possible, to extend the results of Chapter \ref{chap:local-global-principle} to analogues for principal ideal domains, in particular \(\Integers\). The exposition in this chapter follows closely the treatment in \cite{cassels2008rational,gerstein2008basic,jones1950arithmetic}. Chapter \ref{chap:conway-schneeberger} uses the results of the previous chapters to prove the Conway-Schneeberger Fifteen Theorem using the method of escalating lattices. We follow the original exposition in \cite{bhargava2000conway}, as well as \cite{moon2008universal}. We conclude by quickly reviewing the \(290\) theorem of Bhargava and Hanke \cite{bhargava2005universal}, which generalizes the Fifteen Theorem for integral quadratic forms (in the sense of \S\,\ref{sec:integral-forms-def}).