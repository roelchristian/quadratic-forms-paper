\chapter{The fifteen theorem.}
\label{chap:conway-schneeberger}

{\scshape We now} come to the main result of this essay, the Conway-Schneeberger
fifteen theorem. We have earlier established the existence of a natural
bijection between the set of equivalence classes of positive-definite integral
quadratic forms and the set of integral lattices.

\section{Escalation of lattices.}


\section{Escalators of small dimension.}

We can escalate the zero-dimensional lattice uniquely to the lattice generated
by a single vector of norm \(1\). This lattice has matrix form \((1)\) and
correspond to the form \(x^2\). Since \(x^2\) does not represent \(2\), we
obtain the escalator lattice corresponding to the form
\[
    \begin{pmatrix}
        1 & a \\
        a & 2
    \end{pmatrix}  
\]
by the Cauchy-Schwarz inequality, we have \(a^2 \leq 2\) and thus \(a\) is one
of \(0\), \(\pm 1/2\) or \(\pm 1\)

\section{Four-dimensional escalators.}

\section{Five-dimensional escalators and the fifteen theorem.}