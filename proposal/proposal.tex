\documentclass[11pt, letterpaper]{article}


\title{Expository work proposal}
\author{}
\date{}

\RequirePackage{amsmath}
\RequirePackage{amssymb}
\RequirePackage{amsfonts}
\RequirePackage{amsthm}


\RequirePackage[backend=biber]{biblatex}
\RequirePackage[pagestyles]{titlesec}
\RequirePackage{xcolor}
\RequirePackage{titletoc}

\RequirePackage{hyperref}
\hypersetup{
    colorlinks = true,
    linkcolor = green!50!black,
    anchorcolor = green!50!black,
    citecolor = green!50!black,
    filecolor = green!50!black,
    urlcolor = green!50!black
    }

\RequirePackage{graphicx}
\RequirePackage{subcaption}
\renewcommand{\figurename}{Fig.}
\renewcommand{\contentsname}{Contents.}
\renewcommand{\listfigurename}{List of figures.}
\DeclareCaptionFormat{custom}
{%
    {\footnotesize{\lsstylehelp{150}\scshape{{#1#2}}}\textit{#3}}
}
\captionsetup{format=custom}

\RequirePackage{enumitem}

\RequirePackage{tabularx}
\RequirePackage{mathrsfs}

% Only list chapters and sections in the ToC
\setcounter{tocdepth}{1}

% Section format in ToC
\titlecontents{section}
    [0pt]
    {}
    {\quad}
    {}
    {\titlerule*[1pc]{.}\contentspage}



\RequirePackage{fancyhdr}
\RequirePackage{extramarks}
\pagestyle{fancy}
\fancyhf{}
\renewcommand{\headrulewidth}{0pt}
\fancyhead[LE,RO]{\footnotesize\thepage}
\fancyhead[CE]{\scshape \MakeLowercase{\leftmark}}
\fancyhead[CO]{\scshape\MakeLowercase{\rightmark}}

\makeatletter
\renewcommand{\chaptermark}[1]{%
  \markboth{%
    \ifnum\c@secnumdepth>\m@ne
      \@chapapp\ {\small\thechapter}. \ %
    \fi
  #1%
  }{}%
}
\makeatother

\makeatletter
\renewcommand{\chaptermark}[1]{%
        \markboth{\thechapter.\ #1}{}}
% unnumbered chapters in the header
\renewcommand{\chaptermark}[1]{%
        \markboth{#1}{}}
        
\renewcommand{\sectionmark}[1]{\markright{\small #1}}
\renewcommand{\thepage}{\footnotesize\arabic{page}}
\makeatother



\RequirePackage[norule,symbol,perpage]{footmisc}
\RequirePackage{microtype}
\SetTracking{encoding=*, shape=sc}{60}
\widowpenalty10000
\clubpenalty10000
\usepackage{url}
\urlstyle{same}

% \RequirePackage[
%   oldstylenums, largesmallcaps
% ]{kpfonts}



% \usepackage[math-style=ISO]{unicode-math}
% \defaultfontfeatures{Mapping=tex-text}
% \DisableLigatures{encoding = *, shape = sc*}
% \setmainfont[Ligatures={Common,TeX},
% SmallCapsFeatures={
%   LetterSpace=7.5
% }]{Latin Modern}
% \setmathfont[Scale=0.97,Ligatures=TeX,
% math-style=TeX
% ]{Latin Modern Math}
% % \setmathfont[range=it/{Latin,latin}]{Linux Libertine O Italic}
% % % \setmathfont[range=up]{Junicode}
% % % \setmathfont[range=it/{Greek, greek}]{Linux Libertine O Italic}
% % \setmathfont[range=up/{Greek, greek}]{Linux Libertine O}
% % \setmathfont[range=it/{Greek, greek}]{Linux Libertine O Italic}
% \setmathfont[range=frak/{latin,Latin},
%                    Scale=MatchUppercase,
%                    StylisticSet=1,
%                    script-features={},
%                    sscript-features={}
%             ]{Unifraktur Maguntia}

% Theorem environments
\theoremstyle{plain}
\newtheorem*{theorem}{theorem}
\newtheorem*{lemma}{lemma}
\newtheorem*{corollary}{corollary}
% Numbered theorems
\newtheorem{theoremx}{theorem}[subsection]
\newtheorem{lemmax}{lemma}[subsection]
\newtheorem{corollaryx}{corollary}[subsection]
\newtheorem{corollaryb}{corollary}[subsection]



\theoremstyle{definition}
\newtheorem*{definition}{definition}

\renewenvironment{proof}{{\itshape Proof. }}{{\scshape q.e.d.}}

\RequirePackage{longtable}

\RequirePackage{pdfpages}

\AtBeginDocument{
    \let\phi\varphi
    \let\epsilon\varepsilon
    \let\iff\Leftrightarrow
    \let\implies\Rightarrow
    \let\leq\leqslant
    \let\geq\geqslant
    \let\theta\vartheta
    \let\emptyset\varnothing
}

\addbibresource{../ref.bib}

\usepackage{geometry}
\geometry{
    letterpaper,
    left=1in,
    right=1in,
    top=0.75in,
    bottom=0.75in
}

\renewcommand*{\bibfont}{\small}


% Section formatting
\titleformat{\section}
{\normalfont\bfseries}{}{0em}{\normalsize}

\titleformat{name=\section,numberless}[runin]
{\normalfont\bfseries}{}{0em}{\normalsize}

\begin{document}

\thispagestyle{empty}

{\begin{center}
    \large\bfseries
    Expository work proposal    
\end{center}
}


\section*{Title}

On the Conway and Schneeberger fifteen theorem

\section*{Name of student}

Roel Christian G. Yambao

\section{Short description of the topic}

This paper is an exposition of the article ``On the Conway-Schneeberger Fifteen Theorem'' by Manjul Bhargava \cite{bhargava2000conway}. The article provides a proof of the following result originally posited by Conway and Schneeberger in 1993 \cite{conway1999universal,schneeberger1997arithmetic}:
\begin{quote}
    If a positive-definite quadratic form having integer matrix represents every positive integer up to 15 then it represents every positive integer.
\end{quote}

Throughout the paper, the reader will be assumed to have a basic understanding of number theory, linear algebra, and abstract algebra, on the level of the introductory graduate courses on those subjects in De La Salle University. We will begin with a review of the historical development of the theory of quadratic forms and their interest to mathematicians. We will then build the necessary background on quadratic forms, including the theory of lattices, and the reduction theory of quadratic forms, culminating in the proof of the fifteen theorem using escalation of lattices.

\section{References}

The following is an incomplete list of references that will be used in the paper. Additional references may be added to the final version of the paper, as needed.

\nocite{*}
\printbibliography[heading=none]

\section{Timeframe}

\noindent\textbf{October 14 -- November 11.} Gather references; read and summarize the materials; outline and draft the paper.

\noindent\textbf{November 11 -- 20.} Revise and finalize the paper; prepare the presentation.

\noindent\textbf{November 21 -- December 2.} Record the presentation; submission in Canvas.





\end{document}